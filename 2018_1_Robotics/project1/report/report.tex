\documentclass[12pt]{article}
\usepackage[margin=1in]{geometry} 
\usepackage{amsmath,amsthm,amssymb}
\usepackage{xcolor}
\usepackage{listings}
\usepackage[hidelinks]{hyperref}
\lstset{basicstyle=\ttfamily,
  showstringspaces=false,
  commentstyle=\color{red},
  keywordstyle=\color{blue}
}
% \usepackage{enumitem}
\DeclareMathOperator{\A}{\mathbf{A}}
\DeclareMathOperator{\T}{\mathbf{T}}
\DeclareMathOperator{\al}{\mathbf{\alpha}}
\DeclareMathOperator{\Th}{\mathbf{\theta}}

 
\title{Project \#1}
\author{Shao Hua, Huang\\
ICN5534 - Robotics}

\begin{document}
\maketitle
\section{Language and Platform Description}
\begin{enumerate}
  \item
  I use C\texttt{++} to write my program, and use following dependencies.\\
  If possible, please use linux machine to run my code.
  \begin{enumerate}
    \item C\texttt{++} feature is required.
    \item Eigen: C\texttt{++} template library for linear algebra
    \item Qt5: Cross-platform software development for embedded and desktop
  \end{enumerate}
  \item Intall Dependencies\\
  Eigen3: \lstinline{sudo apt install libeigen3-dev}\\
  Qt5: Please refer to \url{https://www.qt.io/download}
  \item Build and Run
  \begin{lstlisting}[language=bash]
mkdir build
cd build
cmake ..
make
./ project1_main_cli 1 <params>
or
./ project1_main_cli 2 <params>
or
./ project1_main_gui
  \end{lstlisting}
\end{enumerate}
\section{Program Architecture}
\begin{enumerate}
  \item project1.hpp: computation functions
    \begin{itemize}
      \item IsValidRange: check if the angle is not NaN and is in valid range
      \item MakeVector6: make a 6 x 1 vector in Eigen
      \item MakeA: given $(d_n, a_n, \al_n, \Th_n)$, make matrix $\A_n$
      \item DoTask1: given $(n, o, a, p)$, compute joint variables
      \item DoTask2: given joint variables, compute $(n, o, a, p)$ and $(x, y, z, \phi, \Th, \phi)$
      \item PrintAnswer: print task1 answer for command line interface
    \end{itemize}
  \item mainwindow.ui, mainwindow.h, mainwindow.cc: GUI codes\\
  \item project1\_main\_cli: CLI version of main\\
  \item project1\_main\_gui: GUI version of main
  \item The implementation is based on the equations derived below, along with some functions in Eigen library.
        See some comments in project1.cc
\end{enumerate}
\section{Equations Derivation}
  In lecture, $\A_n =
  \begin{bmatrix}
  c_n & -s_nc_{\al_n} &  s_ns_{\al_n} & a_nc_n\\
  s_n & -c_nc_{\al_n} & -c_ns_{\al_n} & a_ns_n\\
  0 & s\al_n & c\al_n & d_n\\
  0 & 0 & 0 & 1
  \end{bmatrix}$ where $s_n = \sin(n)$ and $c_n = \cos(n)$\\
  Based on the kinematic table, we have
  \begin{align*}
    \A_1 =
    \begin{bmatrix}
      c_1 & 0 & -s_1 & a_1c_1\\
      s_1 & 0 & c_1 & a_1s_1\\
      0 & -1 & 0 & 0\\
      0 & 0 & 0 & 1
    \end{bmatrix},
    \A_2 =
    \begin{bmatrix}
      c_2 & -s_2 & 0 & a_2c_2\\
      s_2 & c_2 & 0 & a_2s_2\\
      0 & 0 & 1 & 0\\
      0 & 0 & 0 & 1
    \end{bmatrix},
    \A_3 =
    \begin{bmatrix}
      c_3 & -s_3 & 0 & a_3c_3\\
      s_3 & c_3 & 0 & a_3s_3\\
      0 & 0 & 1 & 0\\
      0 & 0 & 0 & 1
    \end{bmatrix}
  \end{align*}
  \begin{align*}
    \A_4 =
    \begin{bmatrix}
      c_4 & 0 & -s_4 & 0\\
      s_4 & 0 & c_4 & 0\\
      0 & -1 & 0 & 0\\
      0 & 0 & 0 & 1
    \end{bmatrix},
    \A_5 =
    \begin{bmatrix}
      c_5 & 0 & s_5 & 0\\
      s_5 & 0 & -c_5 & 0\\
      0 & 1 & 0 & 0\\
      0 & 0 & 0 & 1
    \end{bmatrix},
    \A_6 =
    \begin{bmatrix}
      c_6 & -s_6 & 0 & 0\\
      s_6 & c_6 & 0 & 0\\
      0 & 0 & 1 & 0\\
      0 & 0 & 0 & 1
    \end{bmatrix}
  \end{align*}
  Also calculate inverse of each matrix,\\
  \begin{align*}
    \A_1^{-1} =
    \begin{bmatrix}
      c_1 & s_1 & 0 & -a_1\\
      0 & 0 & -1 & 0\\
      -s_1 & c_1 & 0 & 0\\
      0 & 0 & 0 & 1
    \end{bmatrix},
    \A_2^{-1} =
    \begin{bmatrix}
      c_2 & s_2 & 0 & -a_2\\
      -s_2 & c_2 & 0 & 0\\
      0 & 0 & 1 & 0\\
      0 & 0 & 0 & 1
    \end{bmatrix},
    \A_3^{-1} =
    \begin{bmatrix}
      c_3 & s_3 & 0 & -a_3\\
      -s_3 & c_3 & 0 & 0\\
      0 & 0 & 1 & 0\\
      0 & 0 & 0 & 1
    \end{bmatrix}
  \end{align*}
  \begin{align*}
    \A_4^{-1} =
    \begin{bmatrix}
      c_4 & s_4 & 0 & 0\\
      0 & 0 & -1 & 0\\
      -s_4 & c_4 & 0 & 0\\
      0 & 0 & 0 & 1
    \end{bmatrix},
    \A_5^{-1} =
    \begin{bmatrix}
      c_5 & s_5 & 0 & 0\\
      0 & 0 & 1 & 0\\
      s_5 & -c_5 & 0 & 0\\
      0 & 0 & 0 & 1
    \end{bmatrix},
    \A_6^{-1} =
    \begin{bmatrix}
      c_6 & s_6 & 0 & 0\\
      -s_6 & c_6 & 0 & 0\\
      0 & 0 & 1 & 0\\
      0 & 0 & 0 & 1
    \end{bmatrix}
  \end{align*}
  $\T = \A_1 * \A_2 * \A_3 * \A_4 * \A_5 * \A_6 = 
  \begin{bmatrix}
    n_x & o_x & a_x & p_x\\
    n_y & o_y & a_y & p_y\\
    n_z & o_z & a_z & p_z\\
    0 & 0 & 0 & 1
  \end{bmatrix}$
  
\begin{enumerate}
  \item Calculate $\Th_1$\\
    $$\A_1^{-1}\T = \A_2\A_3\A_4\A_5\A_6$$
    \begin{align*}
      \begin{bmatrix}
        c_1c_2 & s_1c_2 & -s_2 & -a_1c_2-a_2\\
        -c_1s_2 & -s_1s_2 & -c_2 & a_1s_2\\
        -s_1 & c_1 & 0 & 0\\
        0 & 0 & 0 & 1
      \end{bmatrix}\T = 
      \begin{bmatrix}
        \dots & \dots & \dots & a_3c_{23} + a_2c_2\\
        \dots & \dots & \dots & a_3s_{23} + a_2s_2\\
        \dots & \dots & \dots & 0\\
        0 & 0 & 0 & 1
      \end{bmatrix}
    \end{align*}
    That is,
    \begin{align} \label{eq:A1}
      \begin{cases}
        (c_1p_x + s_1p_y - a_1)c_2 - p_zs_2 = a_2 + a_3c_3\\
        p_zc_2 + (c_1p_x + s_1p_y - a_1)s_2 = -a_3s_3\\
        -s_1p_x + c_1p_y = 0
      \end{cases}
    \end{align}
    From third equation of \ref{eq:A1}, we know that
    $$\Th_1 = \arctan(\frac{p_y}{p_x})$$

  \item Calculate $\Th_3$\\
    $$\A_3^{-1}\A_2^{-1}\A_1^{-1}\T = \A_4\A_5\A_6$$
    \begin{align} \label{eq:A3}
      \begin{bmatrix}
        c_1c_{23} & s_1c_{23} & -s_{23} & -a_1c_{23} - a_2c_3 - a_3\\
        -c_1s_{23} & -s_1s_{23} & -c_{23} & a_1s_{23} + a_2s_3\\
        -s_1 & c_1 & 0 & 0\\
        0 & 0 & 0 & 1
      \end{bmatrix}\T = 
      \begin{bmatrix}
        \dots & \dots & c_4s_5 & 0\\
        \dots & \dots & s_4s_5 & 0\\
        -s_5c_6 & s_5s_6 & \dots & 0\\
        0 & 0 & 0 & 1
      \end{bmatrix}
    \end{align}
    From the forth column of the right side of \ref{eq:A3}, we get
    \begin{align} \label{eq:A3_2}
      \begin{cases}
        (c_1p_x + s_1p_y - a_1)c_{23} - p_zs_{23} = a_2c_3 + a_3\\
        p_zc_{23} + (c_1p_x + s_1p_y - a_1)s_{23} = a_2s_3
      \end{cases}
    \end{align}
    Square \ref{eq:A3_2} and sum,
    $$(c_1p_x + s_1p_y - a_1)^2 + p_z^2 = a_2^2 + a_3^2 + 2a_2a_3c_3$$
    Therefore,
    $$\Th_3 = \arccos\left(\frac{(c_1p_x + s_1p_y - a_1)^2 + p_z^2 - a_2^2 - a_3^2}{2a_2a_3c_3}\right)$$

  \item Calculate $\Th_2$\\
    Let
    \begin{align*}
      \begin{cases}
        r\sin(\phi) = p_z\\
        r\cos(\phi) = c_1p_x + s_1p_y - a_1
      \end{cases}
    \end{align*} where
    $$\phi = \arctan(\frac{p_z}{c_1p_x + s_1p_y - a_1})$$\\
    From first two equations of \ref{eq:A1}, we have
    \begin{align*}
      \begin{cases}
        r\cos(\phi)c_2 - r\sin(\phi)s_2 = a_2 + a_3c_3\\
        r\sin(\phi)c_2 + r\cos(\phi)s_2 = -a_3s_3
      \end{cases}
      \Rightarrow
      \begin{cases}
        r\cos(\phi + \Th_2) = a_2 + a_3c_3\\
        r\sin(\phi + \Th_2) = -a_3s_3
      \end{cases}
    \end{align*}
    Therefore,
    \begin{align*}
      \Th_2 & = \phi - \Th_3\\
            & = \arctan(\frac{-a_3s_3}{a_2 + a_3c_3}) - \arctan(\frac{p_z}{c_1p_x + s_1p_y - a_1})
    \end{align*}

  \item Calculate $\Th_4$\\
    From third column of the right side of \ref{eq:A3},
    \begin{align*}
      \begin{cases}
        (c_1a_x + s_1a_y)c_{23} - a_zs_{23} = c_4s_5\\
        a_zc_{23} + (c_1a_x + s_1a_y)s_{23}= -s_4s_5
      \end{cases}
    \end{align*}
    After deviding, we get
    $$\Th_4 = \arctan\left(\frac{a_zc_{23} + (c_1a_x + s_1a_y)s_{23}}{a_zs_{23} - (c_1a_x + s_1a_y)c_{23}}\right)$$

  \item Calculate $\Th_5$\\
    $$\A_4^{-1}\A_3^{-1}\A_2^{-1}\A_1^{-1}\T = \A_5\A_6$$
    \begin{align} \label{eq:A4}
      \begin{bmatrix}
        c_1c_{234} & s_1c_{234} & -s_{234} & \dots\\
        s_1 & -c_1 & 0 & \dots\\
        -c_1s_{234} & -s_1s_{234} & -c_{234} & \dots\\
        0 & 0 & 0 & 1
      \end{bmatrix}\T = 
      \begin{bmatrix}
        \dots & \dots & s_5 & 0\\
        \dots & \dots & -c_5 & 0\\
        \dots & \dots & 0 & 0\\
        0 & 0 & 0 & 1
      \end{bmatrix}
    \end{align}
    From third column of the right side of \ref{eq:A4},
    \begin{align*}
      \begin{cases}
        (c_1a_x + s_1a_y)c_{234} - a_zs_{234} = s_5\\
        s_1a_x - c_1a_y = -c_5
      \end{cases}
    \end{align*}
    Therefore,
    $$\Th_5 = \arctan\left(\frac{(c_1a_x + s_1a_y)c_{234} - a_zs_{234}}{-s_1a_x + c_1a_y}\right)$$

  \item Calculate $\Th_6$\\
    From third row of the right side of \ref{eq:A3},
    \begin{align*}
      \begin{cases}
        -s_1n_x + c_1n_y = -s_5c_6\\
        -s_1o_x + c_1o_y = s_5s_6
      \end{cases}
    \end{align*}
    Therefore,
    $$\Th_6 = \arctan(\frac{s_1o_x - c_1o_y}{-s_1n_x + c_1n_y})$$

\end{enumerate}
\section{Difference Between Algebra and Geometry Approach}
\begin{enumerate}
  \item Algreba Approach\\
    pros:
      \begin{itemize}
        \item equation support, rarely wrong
        \item all solutions provided
      \end{itemize}
    cons:
      \begin{itemize}
        \item complex computation
        \item not intuitive
        \item need to check validity of each angle
      \end{itemize}
  \item Geometry Approach\\
    pros:
      \begin{itemize}
        \item intuitive
        \item simple computation
      \end{itemize}
    cons:
      \begin{itemize}
        \item need to think more for complicate robots
        \item if not consider well, the result may be wrong
      \end{itemize}
\end{enumerate}
\end{document}
