\documentclass[12pt]{article}
\usepackage[margin=1in]{geometry} 
\usepackage{amsmath,amsthm,amssymb}
\usepackage{xcolor}
\usepackage{listings}
\usepackage{cases}
\usepackage[hidelinks]{hyperref}
\renewcommand{\baselinestretch}{1.3}
\DeclareMathOperator{\B}{\mathbf{B}}
\DeclareMathOperator{\C}{\mathbf{C}}
\DeclareMathOperator{\T}{\mathbf{T}}
\DeclareMathOperator{\D}{\Delta}
\lstset{basicstyle=\ttfamily,
  showstringspaces=false,
  commentstyle=\color{red},
  keywordstyle=\color{blue}
}

\title{Homework \#4}
\author{Shao Hua, Huang\\
ICN5534 - Robotics}

\begin{document}
\maketitle
\section{Solve Equations}
\begin{numcases}{}
  a_4t_{acc}^4 + a_3t_{acc}^3 + a_2t_{acc}^2 + a_1t_{acc} + a_0 = \B + \D \C \frac{t_{acc}}{\T}\label{eq1}\\
  a_4t_{acc}^4 - a_3t_{acc}^3 + a_2t_{acc}^2 - a_1t_{acc} + a_0 = \B + \D \B\label{eq2}\\
  4a_4t_{acc}^3 + 3a_3t_{acc}^2 + 2a_2t_{acc} + a_1 = \frac{\D\C}{\T}\label{eq3}\\
  -4a_4t_{acc}^3 + 3a_3t_{acc}^2 - 2a_2t_{acc} + a_1 = -\frac{\D\B}{t_{acc}}\label{eq4}\\
  12a_4t_{acc}^2 + 3a_3t_{acc} + 2a_2 = 0\label{eq5}\\
  12a_4t_{acc}^2 - 3a_3t_{acc} + 2a_2 = 0\label{eq6}
\end{numcases}
\begin{align}
\text{from } \eqref{eq5} \text{ and } \eqref{eq6} & \Rightarrow a_3 = 0\\
\eqref{eq1} - \eqref{eq2} & \Rightarrow a_1 = \frac{1}{2t_{acc}}(\D\C\frac{t_{acc}}{\T}-\D\B) = \frac{1}{2t_{acc}}(\D\C\frac{t_{acc}}{\T}+\D\B)-\frac{1}{t_{acc}}\D\B\\
\eqref{eq3} * 3 & \Rightarrow 12a_4t_{acc}^2 + 6a_2 = \frac{3}{2t_{acc}^2}(\D\C\frac{t_{acc}}{\T}+\D\B)\label{eq7}\\
\eqref{eq7} - \eqref{eq5} & \Rightarrow a_2 = \frac{3}{8t_{acc}^2}(\D\C\frac{t_{acc}}{\T}+\D\B)\label{eq8}\\
\text{sub } \eqref{eq8} \text{ to } \eqref{eq5} & \Rightarrow a_4 = -\frac{1}{16t_{acc}^4}(\D\C\frac{t_{acc}}{\T}+\D\B)\\
\text{sub to } \eqref{eq1} & \Rightarrow a_0 = \frac{3}{16}(\D\C\frac{t_{acc}}{\T}+\D\B) + \B
\end{align}
Let $$h = \frac{t + t_{acc}}{2t_{acc}}$$
Then,
\begin{align*}
q(t) & = (\D\C\frac{t_{acc}}{\T}+\D\B)(-\frac{t^4}{16t_{acc}^4}+\frac{3t^2}{8t_{acc}^2}+\frac{t}{2t_{acc}}+\frac{3}{16})+\B-\frac{t}{t_{acc}}\D\B\\
     & = (\D\C\frac{t_{acc}}{\T}+\D\B)(2-h)h^3 + \B + (1-2h)\D\B\\
\dot{q}(t) & = (\D\C\frac{t_{acc}}{\T}+\D\B)(-\frac{t^3}{4t_{acc}^4}+\frac{3t}{4t_{acc}^2}+\frac{1}{2t_{acc}})-\frac{1}{t_{acc}}\D\B\\
           & = \left[(\D\C\frac{t_{acc}}{\T}+\D\B)(3-2h)h^2-\D\B\right]\frac{1}{t_{acc}}\\
\ddot{q}(t) & = (\D\C\frac{t_{acc}}{\T}+\D\B)(-\frac{3t^2}{4t_{acc}^4}+\frac{3}{4t_{acc}^2})\\
            & = (\D\C\frac{t_{acc}}{\T}+\D\B)(1-h)\frac{3h}{t_{acc}^2}
\end{align*}
Acceleration profile:\\\\\\\\\\\\

\section{Configuration Space Approach}
Paper: A Simple Motion-Planning Algorithm for General Robot Manipulators
\subsection{Brief summary}
There are four main advantages mentioned in this paper
\begin{enumerate}
  \item simple to implement
  \item fast for manipulators with few DOF
  \item can deal with manipulators with many DOF
  \item can deal with cluttered environments and nonconvex polyhedral obstacles
\end{enumerate}
\subsection{Formulate obstacles}
Configuration space (C-space):
\begin{enumerate}
  \item C-space is the space of configurations of a moving object
  \item a configuration is described by a set of joint angles
  \item C-space is described by the robot's joint space
\end{enumerate}
Obstacles:
\begin{enumerate}
  \item map the obstacles in the robot's workspace into its C-space
  \item these C-space obstacles represent those configurations of the moving object that would cause collisions
  \item free space is defined to be the complement of the C-space obstacles
\end{enumerate}
Formulate obstacles:
\begin{enumerate}
  \item let C denote an n-dimensional C-space obstacle for a manipulator with n joint
  \item represent approximations of C by the union of $n-1$ dimensional slice projections (conservative projection)
  \item An approximation of the full obstacle is built as the union of a number of $n-1$ dimensional slice projection
  \item the slice projection can be continued one more step until only zero-dimensional remain
\end{enumerate}
\subsection{Organize the database for the collision-free space}
Use tree of depth $n-1$, where n is the number of joints, and whose branching factor is the number of intervals into which the legal joint parameter range for each joint is divided.
The leaves of the tree are ranges of legal (or forbidden) values for the joint parameter n.
\subsection{Algorithm for finding a feasible path}
A region is made up of linear ranges from a set of adjacent slices such that the ranges all overlap.
The area of common overlap of all the slices in a region is rectangular and called the region's kernel.
We build a region graph where the nodes are regions and the links indicate regions with common boundary.
Path searching is done by an A* search in the region graph from the region containing the start point to the region containing the goal point.
\end{document}
