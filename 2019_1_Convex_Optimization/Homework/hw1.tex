%!TEX program = pdflatex
\documentclass[12pt]{extarticle}
\usepackage[english]{babel}
\usepackage{graphicx}
\usepackage{framed}
\usepackage[normalem]{ulem}
\usepackage{amsmath}
\usepackage{amsthm}
\usepackage{amssymb}
\usepackage{amsfonts}
\usepackage{enumerate}
\usepackage{enumitem}
\usepackage[utf8]{inputenc}
\usepackage[top=1in,bottom=1in,left=1in,right=1in]{geometry}
\usepackage{mdframed}
\usepackage{blindtext}
\usepackage{hyperref}

\hypersetup{
  colorlinks=true,
  linkcolor=blue
}
\theoremstyle{definition}
\newtheorem{exercise}{Exercise}
\everymath{\displaystyle}

\newcommand{\ones}{\mathbf 1}
\newcommand{\E}{\mathop{\bf E{}}}          % not used now
\newcommand{\R}{\mathbb{R}}
\newcommand{\Z}{\mathbb{Z}}
\newcommand{\D}{\mathcal{D}}
\newcommand{\symm}{\mathbb{S}}
\newcommand{\<}{\langle}
\renewcommand{\>}{\rangle}

\newcommand{\nullspace}{{\mathcal N}}      % not used now
\newcommand{\range}{{\mathcal R}}          % not used now
\newcommand{\Rank}{\mathop{\bf Rank}}      % not used now
\newcommand{\tr}{\mathop{\bf tr}}
\newcommand{\diag}{\mathop{\bf diag}}
\newcommand{\card}{\mathop{\bf card}}      % not used now
\newcommand{\rank}{\mathop{\bf rank}}      % not used now
\newcommand{\aff}{\mathop{\bf aff}}
\newcommand{\conv}{\mathop{\bf conv}}
\newcommand{\prox}{\mathbf{prox}}
\newcommand{\prob}{\mathop{\bf prob}}      % not used now
\newcommand{\dist}{\mathop{\bf dist{}}}
\newcommand{\argmin}{\mathop{\rm argmin}}  % not used now
\newcommand{\argmax}{\mathop{\rm argmax}}  % not used now
\newcommand{\epi}{\mathop{\bf epi}}
\newcommand{\Vol}{\mathop{\bf vol}}        % not used now
\newcommand{\dom}{\mathop{\bf dom}}
\newcommand{\intr}{\mathop{\bf int}}
\newcommand{\bd}{\mathop{\bf bd}}
\newcommand{\relintr}{\mathop{\bf relint}}
\newcommand{\cl}{\mathop{\bf cl}}
\newcommand{\sign}{\mathop{\bf sign}}

\newcommand{\cf}{\textit{cf.}}
\newcommand{\eg}{\textit{e.g.}, }
\newcommand{\ie}{\textit{i.e.}, }
\newcommand{\etal}{\textit{et al}. }
\newcommand{\etc}{\textit{etc.}}


\title{Assignment \#1}
\author{Shao Hua, Huang 0750727\\
ECM5901 - Optimization Theory and Application}

\begin{document}
\maketitle

% Exercise 1
\begin{exercise}
  Verify that the $\ell_1$-norm on $\R^n$ defined by $\|x\|_1=\Sigma_{i=1}^n|x_i|$ for $x\in\R^n$ is a norm on $\R^n$. (10\%)
\end{exercise}
\begin{proof}[Verify]
  Check the four properties of a \textit{norm} function.
  \begin{enumerate}[label=(\arabic*)]
    \item nonnegative: $\|x\|_1=\sum_{i=1}^n|x_i|\ge 0\quad\forall x\in\R^n$
    \item definite: $\|x\|_1=\sum_{i=1}^n|x_i|=0\Leftrightarrow x_1=x_2=\dots=x_n=0\Leftrightarrow x=0$
    \item homogeneous: $$\|tx\|_1=\sum_{i=1}^n|tx_i|=\sum_{i=1}^n|t||x_i|=|t|\sum_{i=1}^n|x_i|=|t|\|x\|_1\quad \forall x\in \R^n, t\in\R$$
    \item triangle inequality: $$\|x+y\|_1=\sum_{i=1}^n|x_i+y_i|\le\sum_{i=1}^n(|x_i|+|y_i|)=\|x\|_1+\|y\|_1\quad \forall x,y\in \R^n$$
  \end{enumerate}
  Therefore, $\ell_1$-norm is a \textit{norm}.
\end{proof}

% Exercise 2
\begin{exercise}
  The operator norm on $\R^{m\times n}$ induced by two norms $\|\cdot\|_a$ on $\R^m$ and $\|\cdot\|_b$ on $\R^n$ is defined by
  $$\|X\|_{a,b}=\sup\{\|Xu\|_a\mid\|u\|_b\le 1\}$$
  for $X\in\R^{m\times n}$. Verify that
  $$\|X\|_{1,1}=\max_{j=1,\dots,n}\sum_{i=1}^n|X_{ij}|.$$ (10\%)
\end{exercise}
\newpage
\begin{proof}[Verify]
  Separate the verify into two parts.
  \begin{enumerate}[label=(\alph*)]
    \item $\|X\|_{1,1}\le\max_{j=1,\dots,n}\sum_{i=1}^n|X_{ij}|$\\
      Let $X=[X_1\quad X_2\quad\dots\quad X_n]$ where $X_1,X_2,\dots,X_n$ are column vectors of $X$
      \begin{align*}
        \|X\|_{1,1} &=   \sup\{\|Xu\|_1\mid\|u\|_1\le 1\}\\
                    &=   \sup\left\{\|\sum_{i=1}^nX_iu_i\|_1\mid\|u\|_1\le 1\right\}\\
                    &\le \sup\left\{\sum_{i=1}^n\|X_iu_i\|_1\mid\|u\|_1\le 1\right\}\quad\text{(triangle inequality of norm)}\\
                    &=   \sup\left\{\sum_{i=1}^n\|X_i\|_1|u_i|\mid\|u\|_1\le 1\right\}\quad\text{(homogeneous of norm)}\\
                    &\le \sup\left\{\max\{\|X_1\|_1,\|X_2\|_1,\dots,\|X_n\|_1\}\sum_{i=1}^n|u_i|\mid\|u\|_1\le 1\right\}\\
                    &=   \sup\{\max\{\|X_1\|_1,\|X_2\|_1,\dots,\|X_n\|_1\}\|u\|_1\mid\|u\|_1\le 1\}\\
                    &=   \max\{\|X_1\|_1,\|X_2\|_1,\dots,\|X_n\|_1\}\\
                    &=   \max\left\{\sum_{i=1}^n|X_{i1}|,\sum_{i=1}^n|X_{i2}|,\dots,\sum_{i=1}^n|X_{in}|\right\} = \max_{j=1,\dots,n}\sum_{i=1}^n|X_{ij}|
      \end{align*}
    \item $\|X\|_{1,1}\ge\max_{j=1,\dots,n}\sum_{i=1}^n|X_{ij}|$
      \begin{align*}
        \|X\|_{1,1} = \sup\{\|Xu\|_1\mid\|u\|_1\le 1\} \ge \|Xu\|_1
      \end{align*}
      We choose $u=e_k$ where $k=\arg\max_j\left\{\sum_{i=1}^n|X_{i1}|,\dots,\sum_{i=1}^n|X_{ij}|,\dots,\sum_{i=1}^n|X_{in}|\right\}$\\
      Therefore, $\|u\|_1=1$ is under the constraint, and
      \begin{align*}
        \|X\|_{1,1} &\ge \|Xu\|_1\\
                    &=   |X_{1k}|+|X_{2k}|+\dots+|X_{nk}|\\
                    &=   \sum_{i=1}^n|X_{ik}|\\
                    &=   \max\left\{\sum_{i=1}^n|X_{i1}|,\sum_{i=1}^n|X_{i2}|,\dots,\sum_{i=1}^n|X_{in}|\right\} = \max_{j=1,\dots,n}\sum_{i=1}^n|X_{ij}|
      \end{align*}
    \item From (a) and (b), we know that $\|X\|_{1,1}=\max_{j=1,\dots,n}\sum_{i=1}^n|X_{ij}|$.\qedhere
  \end{enumerate}
\end{proof}

% Exercise 3
\begin{exercise}
  Show that the dual norm of $\ell_1$-norm is the $\ell_\infty$-norm. (10\%)
\end{exercise}
\begin{proof}
  Separate the proof into two parts. ($z, x\in \R^n$)
  \begin{enumerate}[label=(\alph*)]
    \item $\|z\|_\ast\le \|z\|_\infty$
      \begin{align*}
        \|z\|_\ast &=   \sup\{z^Tx\mid\|x\|_1\le 1\}\\
                   &=   \sup\left\{\sum_{i=1}^n(z_ix_i)\mid\|x\|_1\le 1\right\}\\\\
                   &\le \sup\left\{\sum_{i=1}^n(|z_i||x_i|)\mid\|x\|_1\le 1\right\}\\\\
                   &=   \sup\left\{\max\{z_1,z_2,\dots,z_n\}\sum_{i=1}^n|x_i|\mid\|x\|_1\le 1\right\}\\\\
                   &=   \max\{z_1,z_2,\dots,z_n\} = \|z\|_\infty
      \end{align*}
    \item $\|z\|_\ast\ge \|z\|_\infty$
      \begin{align*}
        \|z\|_\ast = \sup\{z^Tx\mid\|x\|_1\le 1\} \ge z^Tx
      \end{align*}
      We choose $x=\sign(z_k)e_k$ where $k=\arg\max_i \{z_1,\dots,z_i,\dots,z_n\}$\\
      Therefore, $\|x\|_1=1$ is under the constraint, and
      \begin{align*}
        \|z\|_\ast&\ge z^Tx\\
                  &=   z^T\sign(z_k)e_k=|z_k|\\
                  &=   \max\{z_1,z_2,\dots,z_n\} = \|z\|_\infty
      \end{align*}
    \item From (a) and (b), we know that $\|z\|_\ast=\|z\|_\infty$.\qedhere
  \end{enumerate}
\end{proof}

% Exercise 4
\begin{exercise}
  The trace of a square matrix $A=[a_{ij}]\in\R^{n\times n}$ is defined as $\tr(A)=\sum_{i=1}^n a_{ii}$. Show that
  \begin{enumerate}[label=(\alph*)]
    \item $\tr(AB) = \tr(BA)$ for $A, B \in \R^{n\times n}$. (5\%)
    \item $\tr(tA+B) = t\tr(A) + \tr(B)$ for $A, B \in \R^{n\times n}$ and $t\in\R$. (5\%)
  \end{enumerate}
\end{exercise}
\begin{proof}
  $ $
  \begin{enumerate}[label=(\alph*)]
    \item $\tr(AB) = \sum_{i=1}^n\left(\sum_{k=1}^na_{ik}b_{ki}\right) = \sum_{k=1}^n\left(\sum_{i=1}^nb_{ki}a_{ik}\right) = \tr(BA)$
    \item $\tr(tA+B) = \sum_{i=1}^n(ta_{ii}+b_{ii}) = t\sum_{i=1}^na_{ii}+\sum_{i=1}^nb_{ii} = t\tr(A)+\tr(B)$ \qedhere
  \end{enumerate}
\end{proof}

% Exercise 5
\begin{exercise}
  Let $\<,\>$ be the inner product on $\R^n$. Prove the Cauchy-Schwarz inequality that for $x, y\in\R^n$, $|\<x,y\>|^2\le\<x,x\>\<y,y\>$. (10\%)
\end{exercise}
\begin{proof}
  We can easily know that $\<x,x\>=|x|^2=\|x\|_2^2$ where $x\in\R^n$.\\
  The proof starts from equation $\|x-\frac{\<x,y\>}{\|y\|_2^2}y\|_2^2\ge 0$, and then we have
  \begin{align*}
    \|x-\frac{\<x,y\>}{\|y\|_2^2}y\|_2^2 
      &= \<x-\frac{\<x,y\>}{\|y\|_2^2}y,x-\frac{\<x,y\>}{\|y\|_2^2}y\>\\
      &= \<x,x\>-2\frac{\<x,y\>}{\|y\|_2^2}\<x,y\>+\frac{\<x,y\>\<x,y\>}{\|y\|_2^4}\<y,y\>\\
      &= \<x,x\>-2\frac{|\<x,y\>|^2}{\|y\|_2^2}+\frac{|\<x,y\>|^2}{\|y\|_2^2}\\
      &= \<x,x\>-\frac{|\<x,y\>|^2}{\|y\|_2^2}\\
      &= \<x,x\>-\frac{|\<x,y\>|^2}{\<y,y\>} \ge 0
  \end{align*}
  Therefore, we can show that $|\<x,y\>|^2\le\<x,x\>\<y,y\>$
\end{proof}

% Exercise 6
\begin{exercise}
  Verify that Frobenious inner product on $\R^{m\times n}$ defined by
  $$\<X,Y\>_F=\tr(X^TY)$$
  for $X, Y\in\R^{m\times n}$ is an inner product. (10\%)
\end{exercise}
\begin{proof}[Verify]
  Check the three properties of \textit{inner product}.
  \begin{enumerate}[label=(\arabic*)]
    \item conjugate symmetry: Since $X, Y\in \R^{m\times n}$, we have $\overline{\<X,Y\>}_F=\<X,Y\>_F$.
      \begin{align*}
        \<X,Y\>_F=\tr(X^TY)=\tr((X^TY)^T)=\tr(Y^TX)=\<Y,X\>_F=\overline{\<Y,X\>}_F
      \end{align*}
    \item linearity: $X, Y, Z\in\R^{m\times n}$, $c\in\R$
      \begin{enumerate}[label=(\roman*)]
        \item $\begin{aligned}[t]\<X+Z,Y\>_F &= \tr((X+Z)^TY)=\tr((X^T+Z^T)Y)=\tr(X^TY+Z^TY)\\
                                             &= \tr(X^TY)+\tr(Z^TY)=\<X,Y\>_F+\<Z,Y\>_F\end{aligned}$
        \item $\<cX,Y\>_F=\tr((cX)^TY)=\tr(cX^TY)=c\ \tr(X^TY)=c\<X,Y\>_F$
      \end{enumerate}
    \item positive-definite: $\<X,X\>_F=\tr(X^TX)=\sum_{i=1}^m\sum_{j=1}^nX_{ij}^2>0$ where $X\in\R^{m\times n}\setminus\{0\}$
  \end{enumerate}
  Therefore, \textit{Frobenious inner product} is an inner product
\end{proof}

% Exercise 7
\begin{exercise}
  Read Appendix A.4. Let $A\in \R^{m\times n}$ and $b\in \R^m$. Consider $f:\R^n\to\R$ defined by $f(x)=\|Ax-b\|_2^2$ for $x\in \R^n$. Show that
  \begin{enumerate}[label=(\alph*)]
    \item $\nabla f(x) = 2A^T(Ax-b)$. (10\%)
    \item $\nabla^2f(x) = 2A^TA$. (10\%)
  \end{enumerate}
\end{exercise}
\begin{proof}[Lemma 1]
  \let\qed\relax
  $Df(x)=A$ where $f(x)=Ax+b$ and $A\in\R^{m\times n}\ \ x,b\in\R^n$\\
  Let $A=[A_1^T\quad A_2^T\quad\dots\quad A_n^T]$ where $A_1,A_2,\dots,A_n$ are row vectors of $A$.\\
  We have $Df(x)_{ij}=\frac{\partial f_i(x)}{\partial x_j}=\frac{\partial (A_ix+b_i)}{\partial x_j}=A_{ij}\Rightarrow Df(x)=A$\\
\end{proof}
\begin{proof}[Lemma 2]
  \let\qed\relax
  $Df(x)=2x^T$ where $f(x)=x^Tx$ and $x\in\R^n$\\
  We have $Df(x)_{i}=\frac{\partial f(x)}{\partial x_i}=\frac{\partial (\sum_{j=1}^n x_j^2)}{\partial x_i}=2x_i\Rightarrow Df(x)=2x^T$\\
\end{proof}
\begin{proof}
  Back to exercise, we have $f(x)=\|Ax-b\|_2^2=(Ax-b)^T(Ax-b)$.\\
  Employ chain rule to prove these equations.
  \begin{enumerate}[label=(\alph*)]
    \item $\nabla f(x)=Df(x)^T=\left(\frac{\partial f(x)}{\partial (Ax-b)}\frac{\partial (Ax-b)}{\partial x}\right)^T=(2(Ax-b)^TA))^T=2A^T(Ax-b)$\\
    \item $\nabla^2 f(x)=D\nabla f(x)=\frac{\partial (2A^T(Ax-b))}{\partial x}=\frac{\partial (2A^TAx-2A^Tb)}{\partial x}=2A^TA$ \qedhere
  \end{enumerate}
\end{proof}

% Exercise 8
\begin{exercise}
  For a symmetric matrix $A\in\R^{n\times n}$, it has an eigenvalue (spectral) decomposition
  $$A=Q\Lambda Q^T,$$
  where $Q\in\R^{n\times n}$ is an orthogonal matrix such that $Q^TQ=QQ^T=I$ and $\Lambda=\diag(\lambda_1,\dots,\lambda_n)$ is a diagonal matrix with entries that are eigenvalues of $A$.
  \begin{enumerate}[label=(\alph*)]
    \item Show that $\tr(A)=\sum_{i=1}^n\lambda_i$. (5\%)
    \item Show that $x^TAx\ge 0$ for any $x\in\R^n$ if and only if $\lambda_i\ge 0$ for $i=1,\dots,n$. ($A$ is called positive semidefinite if all the eigenvalues $\lambda_i$ are nonnegative.) (10\%)
    \item Let $\lambda_{\max}=\max\{\lambda_1,\dots,\lambda_n\}$ and $\lambda_{\min}=\min\{\lambda_1,\dots,\lambda_n\}$. Show that
          $$\lambda_{\min}x^Tx\le x^TAx\le \lambda_{\max}x^Tx.$$
          (10\%)
  \end{enumerate}
\end{exercise}
\newpage
\begin{proof}
  $ $
  \begin{enumerate}[label=(\alph*)]
    \item eigenvalues are roots of the characteristic equation, \ie roots of
      \begin{align}
        \det(A-\lambda I)
          & = \begin{vmatrix}
                (a_{11}-\lambda) & a_{12} & \cdots & a_{1n}\\
                a_{21} & (a_{22}-\lambda) & \cdots & a_{2n}\\
                \vdots & \vdots & \ddots & \vdots \\
                a_{n1} & a_{n2} & \cdots & (a_{nn} - \lambda)
              \end{vmatrix}\nonumber\\
          & = (-1)^n\lambda^n+(-1)^{n-1}\left(\sum_{i=1}^na_{ii}\right)\lambda^{n-1}+\dots+\det(A)\nonumber\\
          & = (-1)^n\lambda^n+(-1)^{n-1}\tr(A)\lambda^{n-1}+\dots+\det(A)\label{eq:8.1}\\
          & = 0\nonumber
      \end{align}
      We can rewrite $\det(A-\lambda I)=0$ by its roots $\lambda_1,\dots,\lambda_n$ with the information of term $(-1)^n\lambda^n$.
      \begin{align}
        \det(A-\lambda I)
          & = (\lambda_1-\lambda)(\lambda_2-\lambda)\dots(\lambda_n-\lambda)\nonumber\\
          & = (-1)^n\lambda^n+(-1)^{n-1}\left(\sum_{i=1}^n\lambda_i\right)\lambda^{n-1}+\dots+\prod_{i=1}^n\lambda_i\label{eq:8.2}
      \end{align}
      Compare \eqref{eq:8.1} with \eqref{eq:8.2}, we know that $\tr(A)=\sum_{i=1}^n\lambda_i$.      
    \item Separate the proof into two parts.
      \begin{enumerate}[label=(\roman*)]
        \item Proposition of $\rightarrow$\\
          Choose $x$ to be eigenvectors of $A$, \ie $v_i, i=1,2,\dots,n$\\
          Then $Av_i=\lambda_iv_i$ holds where $\lambda_i$ represents corresponding eigenvalue
          \begin{align*}
            x^TAx=v_i^TAv_i=v_i^T\lambda_iv_i=\lambda_i\|v_i\|_2^2\ge 0 \Rightarrow \lambda_i \ge 0
          \end{align*}
          $x^TAx\ge 0$ for any $x\in\R^n\Rightarrow\lambda_i\ge 0$ for $i=1,\dots,n$.
        \item Proposition of $\leftarrow$\\
          Let $y=Q^Tx$, \ie $x=Qy$
          \begin{align*}
            x^TAx=(Qy)^TA(Qy)=y^TQ^TQ\Lambda Q^TQy=y^T\Lambda y=\sum_{i=1}^n\lambda_iy_i^2
          \end{align*}
          We know that all $\lambda_i\ge 0$; therefore, $x^TAx\ge 0$\\
          $x^TAx\ge 0$ for any $x\in\R^n\Leftarrow\lambda_i\ge 0$ for $i=1,\dots,n$.
        \item From (i) and (ii), we get\\
        $x^TAx\ge 0$ for any $x\in\R^n\Leftrightarrow\lambda_i\ge 0$ for $i=1,\dots,n$.
      \end{enumerate}
\newpage
    \item Separate the proof into two parts. Similar to (b)(ii), let $y=Q^Tx$, \ie $x=Qy$.
      \begin{enumerate}[label=(\roman*)]
        \item $\lambda_{\min}x^Tx\le x^TAx$
          \begin{align*}
            x^TAx &= (Qy)^TA(Qy)=y^TQ^TQ\Lambda Q^TQy=y^T\Lambda y=\sum_{i=1}^n\lambda_iy_i^2\\
            \lambda_{\min}x^Tx &=\lambda_{\min}(Qy)^T(Qy)=\lambda_{\min}y^TQ^TQy=\lambda_{\min}y^Ty=\lambda_{\min}\sum_{i=1}^ny_i^2
          \end{align*}
          It is easy to show that $\lambda_{\min}\sum_{i=1}^ny_i^2\le\sum_{i=1}^n\lambda_iy_i^2\Rightarrow\lambda_{\min}x^Tx\le x^TAx$.
        \item $x^TAx\le \lambda_{\max}x^Tx$
          \begin{align*}
            \lambda_{\max}x^Tx =\lambda_{\max}(Qy)^T(Qy)=\lambda_{\max}y^TQ^TQy=\lambda_{\max}y^Ty=\lambda_{\max}\sum_{i=1}^ny_i^2
          \end{align*}
          Similar to (i), $\sum_{i=1}^n\lambda_iy_i^2\le\lambda_{\max}\sum_{i=1}^ny_i^2\Rightarrow x^TAx\le \lambda_{\max}x^Tx$.
        \item From (i) and (ii), we get $\lambda_{\min}x^Tx\le x^TAx\le \lambda_{\max}x^Tx$.\qedhere      
      \end{enumerate}
  \end{enumerate}
\end{proof}

\end{document}
