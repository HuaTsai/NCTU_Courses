%!TEX program = pdflatex
\documentclass[12pt]{extarticle}
\usepackage[english]{babel}
\usepackage{graphicx}
\usepackage{framed}
\usepackage[normalem]{ulem}
\usepackage{amsmath}
\usepackage{amsthm}
\usepackage{amssymb}
\usepackage{amsfonts}
\usepackage{enumerate}
\usepackage{enumitem}
\usepackage[utf8]{inputenc}
\usepackage[top=1in,bottom=1in,left=1in,right=1in]{geometry}
\usepackage{mdframed}
\usepackage{blindtext}
\usepackage{hyperref}

\hypersetup{
  colorlinks=true,
  linkcolor=blue
}
\theoremstyle{definition}
\newtheorem{exercise}{Exercise}
\everymath{\displaystyle}

\newcommand{\ones}{\mathbf 1}
\newcommand{\E}{\mathop{\bf E{}}}          % not used now
\newcommand{\R}{\mathbb{R}}
\newcommand{\Z}{\mathbb{Z}}
\newcommand{\D}{\mathcal{D}}
\newcommand{\symm}{\mathbb{S}}
\newcommand{\<}{\langle}
\renewcommand{\>}{\rangle}

\newcommand{\nullspace}{{\mathcal N}}      % not used now
\newcommand{\range}{{\mathcal R}}          % not used now
\newcommand{\Rank}{\mathop{\bf Rank}}      % not used now
\newcommand{\tr}{\mathop{\bf tr}}
\newcommand{\diag}{\mathop{\bf diag}}
\newcommand{\card}{\mathop{\bf card}}      % not used now
\newcommand{\rank}{\mathop{\bf rank}}      % not used now
\newcommand{\aff}{\mathop{\bf aff}}
\newcommand{\conv}{\mathop{\bf conv}}
\newcommand{\prox}{\mathbf{prox}}
\newcommand{\prob}{\mathop{\bf prob}}      % not used now
\newcommand{\dist}{\mathop{\bf dist{}}}
\newcommand{\argmin}{\mathop{\rm argmin}}  % not used now
\newcommand{\argmax}{\mathop{\rm argmax}}  % not used now
\newcommand{\epi}{\mathop{\bf epi}}
\newcommand{\Vol}{\mathop{\bf vol}}        % not used now
\newcommand{\dom}{\mathop{\bf dom}}
\newcommand{\intr}{\mathop{\bf int}}
\newcommand{\bd}{\mathop{\bf bd}}
\newcommand{\relintr}{\mathop{\bf relint}}
\newcommand{\cl}{\mathop{\bf cl}}
\newcommand{\sign}{\mathop{\bf sign}}

\newcommand{\cf}{\textit{cf.}}
\newcommand{\eg}{\textit{e.g.}, }
\newcommand{\ie}{\textit{i.e.}, }
\newcommand{\etal}{\textit{et al}. }
\newcommand{\etc}{\textit{etc.}}


\title{Assignment \#4}
\author{Shao Hua, Huang 0750727\\
ECM5901 - Optimization Theory and Application}
\begin{document}
\maketitle

% Exercise 1
\begin{exercise}
  For a subspace $V$ of $\R^n$, its orthogonal complement, denoted by $V^\perp$, is defined as $V^\perp=\{x\mid \<x,z\>=0\;\text{for all}\;z\in V\}$.
  Suppose that $A\in\R^{m\times n}$. Show that
  \begin{align*}
    \nullspace(A) = (\range(A^T))^\perp.
  \end{align*}
  where $\nullspace(A)$ is the null space of $A$ and $\range(A^T)$ is the range of $A^T$. (15\%)
\end{exercise}
\begin{proof}
  The definition of range is $\range(A)=\{Ax\mid x\in\R^n\}$, and the definition of nullspace is $\nullspace(A)=\{x\in\R^n\mid Ax=0\}$.
  Let $A=[a_1\;a_2\;\cdots\;a_m]^T$ where $a_i$ are row vectors of $A$, and $A^T=[a_1\;a_2\;\cdots\;a_m]$ where $a_i$ are column vectors of $A^T$.\par
  Separate the proof into two parts.
  \begin{enumerate}[label=(\roman*)]
    \item $\nullspace(A)\subseteq (\range(A^T))^\perp$, \ie $v\in\nullspace(A)\Rightarrow v\in(\range(A^T))^\perp$
          \begin{align*}
            v\in\nullspace(A)&\Rightarrow Av=0\\
                             &\Rightarrow a_i^Tv=0,\;i=1,\dots,m\\
                             &\Rightarrow c_1a_1^Tv+c_2a_2^Tv+\cdots+c_ma_m^Tv=0\\
                             &\Rightarrow w^Tv=0\quad\forall\;w=c_1a_1+c_2a_2+\cdots+c_ma_m=A^Tc\\
                             &\Rightarrow w^Tv=0\quad\forall\;w\in\range(A^T)\\
                             &\Rightarrow v\in(\range(A^T))^\perp
          \end{align*}
          where $c=(c_1,c_2,\dots,c_m)\in\R^m$
    \item $\nullspace(A)\supseteq (\range(A^T))^\perp$, \ie $v\in(\range(A^T))^\perp\Rightarrow v\in\nullspace(A)$
          \begin{align*}
            v\in(\range(A^T))^\perp&\Rightarrow w^Tv=0\quad\forall\;w\in\range(A^T)\\
                                   &\Rightarrow w^Tv=0\quad\forall\;w=A^Tc=c_1a_1+c_2a_2+\cdots+c_ma_m\\
                                   &\Rightarrow a_i^Tv=0,\;i=1,\dots,m\\
                                   &\Rightarrow Av=0
          \end{align*}
          where choose $c=e_1,e_2,\dots,e_m$
    \item From (i) and (ii), we know that $\nullspace(A)=(\range(A^T))^\perp$\qedhere
  \end{enumerate}
\end{proof}

% Exercise 2
\begin{exercise}
  Prove that $x^\ast=(1,1/2,-1)$ is optimal for the optimization problem
  \begin{align*}
    \text{minimize\;\;}\quad& f_0(x)=(1/2)x^TPx+q^Tx+r\\
    \text{subject to}  \quad& -1\le x_i \le 1,\;i=1,2,3
  \end{align*}
  where
  \begin{align*}
    P= \begin{bmatrix}13&12&-2\\12&17&6\\-2&6&12\end{bmatrix},\;
    q= \begin{bmatrix}-22\\-14.5\\13\end{bmatrix},\;
    r= 1.
  \end{align*}
  (20\%)
\end{exercise}
\begin{proof}
  First check whether $f_0(x)$ is convex. Since $f_0$ is twice differentiable and $P\in\symm^3$, \ie check $\nabla^2f_0(x)=P\succeq 0$.
  After some calculations, eigenvalues of $P$ are $(\lambda_1,\lambda_2,\lambda_3)\approx(27.898,13.843,0.259)\succeq 0$.
  Therefore, $P$ is positive semidefinite.\par
  Since $f_0$ is convex and inequality constraint functions are convex (intersection of halfspaces), the original problem becomes a convex optimization problem.
  Thus we can apply equation (4.21) in \S 4.2.3 to check whether $x^\ast=(1,1/2,-1)$ is optimal.
  \begin{align}
    \nabla f_0(x^\ast)^T(y-x^\ast) &=
        \begin{bmatrix}-1&0&2\end{bmatrix}
        \begin{bmatrix}y_1-1\\y_2-1/2\\y_3+1\end{bmatrix}\nonumber\\
      &= -1(y_1-1)+2(y_3+1)\ge 0\label{eq:2.1}
  \end{align}
  Since inequality \eqref{eq:2.1} holds for all $y$ in the feasible set, $x^\ast$ is proved to be an optimal.
\end{proof}

% Exercise 3
\begin{exercise}
  Consider a problem of the form
  \begin{align}
    \text{minimize\;\;}\quad& f_0(x)/(c^Tx+d)\nonumber\\
    \text{subject to}  \quad& f_i(x)\le 0,\;i=1,\dots,m\label{eq:3.1}\\
                            & Ax=b\nonumber
  \end{align}
  where $f_0,f_1,\dots,f_m$ are convex, and the domain of the objective function is defined as $\{x\in\dom f_0\mid c^Tx+d>0\}$.
  \begin{enumerate}[label=(\alph*)]
    \item Show that the problem is equivalent to
          \begin{align}
            \text{minimize\;\;}\quad& g_0(y,t)\nonumber\\
            \text{subject to}  \quad& g_i(y,t)\le 0,\;i=1,\dots,m\label{eq:3.2}\\
                                    & Ay = bt\nonumber\\
                                    & c^Ty+dt=1\nonumber
          \end{align}
          where $g_i$ is the perspective of $f_i$ (see \S 3.2.6).
          The variables are $y\in\R^n$ and $t\in\R$. (15\%)
    \item Show that this problem is convex. (10\%)
  \end{enumerate}
\end{exercise}
\begin{proof}
  We have $g_i(y,t)=tf_i(y/t)$ by definition of perspective function.
  \begin{enumerate}[label=(\alph*)]
    \item Separate the proof into two parts.
      \begin{enumerate}[label=(\roman*)]
        \item If $x$ is feasible in \eqref{eq:3.1}, then $(y, t)$ is feasible in \eqref{eq:3.2}.\\
              Define $y=\frac{x}{c^Tx+d}$ and $t=\frac{1}{c^Tx+d}>0$ (since $x\in\dom f_0$), then
              \begin{align*}
                \begin{cases}
                  (y,t)\in\dom g_i=\{(y,t)\mid y/t\in\dom f_i,\;t>0\},\;i=1,\dots,m\\
                  g_i(y,t)=tf_i(y/t)=tf_i(x)\le 0,\;i=1,\dots,m\\
                  Ay=\frac{Ax}{c^Tx+d}=\frac{b}{c^Tx+d}=bt\\
                  c^Ty+dt=\frac{c^Tx+d}{c^Tx+d}=1
                \end{cases}
              \end{align*}
              Thus $(y,t)$ is feasible with $g_0(y,t)=f_0(x)/(c^Tx+d)$.
        \item If $(y, t)$ is feasible in \eqref{eq:3.2}, then $x$ is feasible in \eqref{eq:3.1}.\\
              Define $x=y/t$ ($t>0$ by definition of the perspective function), then
              \begin{align*}
                \begin{cases}
                  (y,t)\in\dom g_i=\{(y,t)\mid y/t\in\dom f_i,\;t>0\}\Rightarrow x\in\dom f_i,\;i=1,\dots,m\\
                  f_i(x)=f_i(y/t)=g_i(y,t)/t\le 0,\;i=1,\dots,m\\
                  Ax=\frac{Ay}{t}=b
                \end{cases}
              \end{align*}
              Thus $x$ is feasible with $f_0(x)/(c^Tx+d)=g_0(y,t)/(c^Txt+dt)=g_0(y,t)$.
        \item From (i) and (ii), we know that problem \eqref{eq:3.1} is equivalent to problem \eqref{eq:3.2}.
      \end{enumerate}
    \item As discussed in \S 3.2.6, the perspective operation preserves convexity, \ie $g_i(y,t)$ is convex since $f_i(x)$ is convex where $i=0,\dots,m$.
          In conclusion, the problem \eqref{eq:3.2} have a convex objective function, $m$ convex inequality constraint functions, and the row mumber of $A$ plus $1$ affine equality constraint functions.
          Therefore, \eqref{eq:3.2} is a convex optimization problem.\qedhere
  \end{enumerate}
\end{proof}

% Exercise 4
\begin{exercise}
  (Network flow problem.) Consider a network of n nodes, with directed links connecting each pair of nodes.
  The variables in the problem are the flows on each link: $x_{ij}$ will denote the flow from node $i$ to node $j$.
  The cost of the flow along the link from node $i$ to node $j$ is given by $c_{ij}x_{ij}$, where $c_ij$ are given constants.
  The total cost across the network is
  \begin{align*}
    C=\sum_{i,j=1}^nc_{ij}x_{ij}
  \end{align*}
  Each link flow $x_{ij}$ is also subject to a given lower bound $l_{ij}$ (usually assumed to be nonnegative) and an upper bound $u_{ij}$.
  The external supply at node $i$ is given by $b_i$, where $b_i>0$ means an external flow enters the network at node $i$, and $b_i<0$ means that at node $i$, an amount $|b_i|$ flows out of the network.
  We assume that $\sum_{i}b_i=0$, \ie the total external supply equals total external demand.
  At each node we have conservation of flow: the total flow into node $i$ along links and the external supply, minus the total flow out along the links, equals zero.
  The problem is to minimize the total cost of flow through the network, subject to the constraints described above.
  Formulate this problem as a linear program. (20\%)
\end{exercise}
\begin{proof}[Solution]
  \let\qed\relax
  The problem can be formulated by linear programming
  \begin{align*}
    \text{minimize\;\;}\quad& C=\sum_{i,j=1}^nc_{ij}x_{ij}\\
    \text{subject to}  \quad& b_i+\sum_{j=1}^nx_{ji}-\sum_{j=1}^nx_{ij}=0,\;i=1,\dots,n\\
                            & l_{ij}\le x_{ij}\le u_{ij}
  \end{align*}
\end{proof}

% Exercise 5
\begin{exercise}
  Give an explicit solution of the following \textbf{QCQP}s. (Minimizing a linear function over an ellipsoid)
  \begin{align*}
    \text{minimize\;\;}\quad& c^Tx\\
    \text{subject to}  \quad& (x-x_c)^TA(x-x_c)\le 1
  \end{align*}
  where $A\in\symm_{++}^n$ and $c\neq 0$. (20\%)
\end{exercise}
\begin{proof}[Solution]
  \let\qed\relax
  Let $y=A^{1/2}(x-x_c)$, \ie $x=A^{-1/2}y+x_c$, then the problem becomes
  \begin{align*}
    \text{minimize\;\;}\quad& f_0(x)=c^TA^{-1/2}y+c^Tx_c\\
    \text{subject to}  \quad& y^Ty\le 1
  \end{align*}
  We have $\nabla f_0(x)=A^{1/2}c$, and the solution is $y^\ast=-\frac{A^{1/2}c}{\|A^{1/2}c\|_2}$, \ie
  \begin{align*}
    x^\ast=x_c-\frac{A^{-1}}{\|A^{1/2}c\|_2}c
  \end{align*}
\end{proof}

\end{document}
