%!TEX program = pdflatex
\documentclass[12pt]{extarticle}
\usepackage[english]{babel}
\usepackage{graphicx}
\usepackage{framed}
\usepackage[normalem]{ulem}
\usepackage{amsmath}
\usepackage{amsthm}
\usepackage{amssymb}
\usepackage{amsfonts}
\usepackage{enumerate}
\usepackage{enumitem}
\usepackage[utf8]{inputenc}
\usepackage[top=1in,bottom=1in,left=1in,right=1in]{geometry}
\usepackage{mdframed}
\usepackage{blindtext}
\usepackage{xcolor}
\usepackage{listings}
\usepackage[numbered,framed]{matlab-prettifier}
\usepackage{hyperref}

\hypersetup{
  colorlinks=true,
  linkcolor=blue
}
\theoremstyle{definition}
\newtheorem{exercise}{Exercise}
\everymath{\displaystyle}

\newcommand{\ones}{\mathbf 1}
\newcommand{\E}{\mathop{\bf E{}}}          % not used now
\newcommand{\R}{\mathbb{R}}
\newcommand{\Z}{\mathbb{Z}}
\newcommand{\D}{\mathcal{D}}
\newcommand{\symm}{\mathbb{S}}
\newcommand{\<}{\langle}
\renewcommand{\>}{\rangle}

\newcommand{\nullspace}{{\mathcal N}}      % not used now
\newcommand{\range}{{\mathcal R}}          % not used now
\newcommand{\Rank}{\mathop{\bf Rank}}      % not used now
\newcommand{\tr}{\mathop{\bf tr}}
\newcommand{\diag}{\mathop{\bf diag}}
\newcommand{\card}{\mathop{\bf card}}      % not used now
\newcommand{\rank}{\mathop{\bf rank}}      % not used now
\newcommand{\aff}{\mathop{\bf aff}}
\newcommand{\conv}{\mathop{\bf conv}}
\newcommand{\prox}{\mathbf{prox}}
\newcommand{\prob}{\mathop{\bf prob}}      % not used now
\newcommand{\dist}{\mathop{\bf dist{}}}
\newcommand{\argmin}{\mathop{\rm argmin}}  % not used now
\newcommand{\argmax}{\mathop{\rm argmax}}  % not used now
\newcommand{\epi}{\mathop{\bf epi}}
\newcommand{\Vol}{\mathop{\bf vol}}        % not used now
\newcommand{\dom}{\mathop{\bf dom}}
\newcommand{\intr}{\mathop{\bf int}}
\newcommand{\bd}{\mathop{\bf bd}}
\newcommand{\relintr}{\mathop{\bf relint}}
\newcommand{\cl}{\mathop{\bf cl}}
\newcommand{\sign}{\mathop{\bf sign}}

\newcommand{\cf}{\textit{cf.}}
\newcommand{\eg}{\textit{e.g.}, }
\newcommand{\ie}{\textit{i.e.}, }
\newcommand{\etal}{\textit{et al}. }
\newcommand{\etc}{\textit{etc.}}


\title{Assignment \#5}
\author{Shao Hua, Huang 0750727\\
ECM5901 - Optimization Theory and Application}
\begin{document}
\maketitle

% Exercise 1
\begin{exercise}
  Let $A_0=\begin{bmatrix}10&8&12&15&15\\8&14&8&7&9\\12&8&10&13&9\\15&7&13&4&10\\15&9&9&10&4\end{bmatrix}$,
      $A_1=\begin{bmatrix}12&11&14&10&3\\11&14&10&14&6\\14&10&16&18&4\\10&14&18&18&8\\3&6&4&8&8\end{bmatrix}$,\\
      $A_2=\begin{bmatrix}4&13&12&16&6\\13&4&14&9&15\\12&14&6&5&5\\16&9&5&2&6\\6&15&5&6&8\end{bmatrix}$\\
  Suppose $A:\R^2\rightarrow\symm^5$ is defined by
  \begin{align*}
    A(x)=A_0+x_1A_1+x_2A_2.
  \end{align*}
  Let $\lambda_1(x)\ge\lambda_2(x)\ge\lambda_3(x)\ge\lambda_4(x)\ge\lambda_5(x)$ denoted the eigenvalues of $A(x)$.
  \begin{enumerate}[label=(\alph*)]
    \item Formulate the problem of minimizing the spread of the eigenvalues $\lambda_1(x)-\lambda_5(x)$ as an SDP. (15\%)
    \item Solve (a) by using \textsc{Matlab} with the CVX tool. What are the optimal point and optimal value? (25\%)
  \end{enumerate}
\end{exercise}
\begin{proof}[Solution]
  \let\qed\relax
  $ $
  \begin{enumerate}[label=(\alph*)]
    \item Introduce additional variable $t=(t_1,t_2)$.
          We can use the property that $\lambda_i(x)\le s$ ($\lambda_i(x)\ge s$) if and only if $A(x)\preceq sI$ ($A(x)\succeq sI$), then we have
          \begin{align*}
            \begin{cases}
              \lambda_1(x)\le t_1\;\text{if and only if}\;A(x)\preceq t_1I\\
              -\lambda_5(x)\le -t_2\;\text{if and only if}\;A(x)\succeq t_2I
            \end{cases}
          \end{align*}
          Therefore, the problem to minimize $\lambda_1(x)-\lambda_5(x)$ becomes
          \begin{align*}
            \text{minimize\;\;}\quad& t_1-t_2\\
            \text{subject to}  \quad& t_2I\preceq A(x)\preceq t_1I
          \end{align*}
          It is a semidefinite program (SDP).
    \item \textsc{Matlab} code

\begin{lstlisting}[style=Matlab-editor]
% file: hw5_1.m
% assign matrices A0, A1, and A2

cvx_begin sdp quiet
  variables x(2) t(2)
  minimize(t(1)-t(2))
  t(2) * eye(5) <= A0 + x(1) * A1 + x(2) * A2
  A0 + x(1) * A1 + x(2) * A2 <= t(1) * eye(5)
cvx_end

disp(['Optimal value: ', sprintf('%f', cvx_optval)]);
disp('Optimal point:');
disp(['  x = [ ', sprintf('%f ', x ), ']']);
disp(['  t = [ ', sprintf('%f ', t ), ']']);
\end{lstlisting}
Result
\begin{lstlisting}[style=Matlab-editor]
>> run hw5_1.m
Optimal value: 28.154385
Optimal point:
  x = [ -0.596605 -0.335843 ]
  t = [ 13.759723 -14.394662 ]
\end{lstlisting}

  \end{enumerate}
\end{proof}

% Exercise 2
\begin{exercise}
  Consider the optimization problem
  \begin{align*}
    \text{minimize\;\;}\quad& x^2+1\\
    \text{subject to}  \quad& (x+1)(x+4)\le 0
  \end{align*}
  with variable $x\in\R$.
  \begin{enumerate}[label=(\alph*)]
    \item (Analysis of primal problem.) Give the feasible set, the optimal value, and the optimal solution. (5\%)
    \item Derive the Lagrange dual function $g$. (5\%)
    \item State the dual problem, and verify that it is a concave maximization problem. (5\%)
    \item Find the dual optimal value and dual optimal solution? Does the strong duality hold? (5\%)
  \end{enumerate}
\end{exercise}
\begin{proof}[Solution]
  \let\qed\relax
  $ $
  \begin{enumerate}[label=(\alph*)]
    \item The objective $f_0(x)=x^2+1$ is a concave upward parabola with minimization value at point $x=0$.
          From the constraint function $f_1(x)=(x+1)(x+4)\le 0$, we know the feasible set is the interval $[-4,-1]$.
          Therefore, the optimal point is at $x^\ast=-1$ with the optimal value $p^\ast=(x^\ast)^2+1=2$.
    \item The Lagrange is
          \begin{align*}
            L(x,\lambda)=f_0(x)+\lambda f_1(x)=(x^2+1)+\lambda(x+1)(x+4)=(\lambda+1)x^2+5\lambda x+(4\lambda+1)
          \end{align*}
          The Lagrange dual function is
          \begin{align*}
            g(\lambda)&=\inf_{x\in\D}L(x,\lambda)\\
                      &=\inf_{x\in\D}\left((\lambda+1)x^2+5\lambda x+(4\lambda+1)\right)\\
                      &=
              \begin{cases}
                4\lambda+1-\frac{25\lambda^2}{4\lambda+4}, &\lambda>-1\\
                -\infty, &\lambda\le -1
              \end{cases}
          \end{align*}
          where we compute the optimal value of parabola by the formula $(4ac-b^2)/4a$.
    \item The Lagrange dual problem is
          \begin{align*}
            \text{maximize\;}\quad& g(\lambda)=
              \begin{cases}
                4\lambda+1-\frac{25\lambda^2}{4\lambda+4}, &\lambda>-1\\
                -\infty, &\lambda\le -1
              \end{cases}\\
            \text{subject to}  \quad& \lambda\ge 0
          \end{align*}
          With $\lambda\ge 0$ and abuse of the terminology, the Lagrange dual problem is
          \begin{align*}
            \text{maximize\;}\quad& g(\lambda)=4\lambda+1-\frac{25\lambda^2}{4\lambda+4}\\
            \text{subject to}  \quad& \lambda\ge 0
          \end{align*}
          $f(\lambda)$ is a concave function since
          \begin{enumerate}[label=(\roman*)]
            \item $4\lambda+1$ is an affine function
            \item $\frac{25\lambda^2}{4\lambda+4}$ is affine mapping of the quadratic-over-linear function, thus convex
            \item $g(\lambda)$ is an affine function minus a convex function, thus concave
          \end{enumerate}
          The Lagrange dual problem is a concave maximization problem.
    \item Find optimal point by the derivative of $g$
          \begin{align*}
            g'(\lambda)|_{\lambda=\lambda^\ast}=4-\frac{25\lambda^\ast(\lambda^\ast+2)}{4(\lambda^\ast+1)^2}=0\Rightarrow \lambda^\ast=\frac{2}{3},-\frac{8}{3}\;\text{(invalid)}
          \end{align*}
          Then the optimal value is
          \begin{align*}
            d^\ast=g(\lambda^\ast)=g(\frac{2}{3})=\frac{8}{3}+1-\frac{\frac{100}{9}}{\frac{8}{3}+4}=2
          \end{align*}
          Therefore the strong duality $d^\ast=p^\ast$ holds.
  \end{enumerate}
\end{proof}

% Exercise 3
\begin{exercise}
  (Dual of general LP). Find the dual function of the LP
  \begin{align*}
    \text{minimize\;\;}\quad& c^Tx\\
    \text{subject to}  \quad& Gx\preceq h\\
                            & Ax=b
  \end{align*}
  Give the dual problem, and make the implicit equality constraints explicit. (20\%)
\end{exercise}
\begin{proof}[Solution]
  \let\qed\relax
  $ $
  \begin{enumerate}[label=(\alph*)]
    \item Lagrange dual function\\
          The Lagrange is 
          \begin{align*}
            L(x,\lambda,\nu)&=f_0(x)+\sum_{i=1}^m\lambda_if_i(x)+\sum_{i=1}^p\nu_ih_i(x)\\
                            &=c^Tx+\lambda^T(Gx-h)+\nu^T(Ax-b)\\
                            &=-\lambda^Th-\nu^Tb+(c+G^T\lambda+A^T\nu)^Tx
          \end{align*}
          where $h_i$ is equality constraint function and unrelated to $h$ in the origin problem.\\
          The Lagrange dual function is
          \begin{align*}
            g(\lambda,\nu)&=\inf_{x\in\D}L(x,\lambda,\nu)\\
                          &=-\lambda^Th-\nu^Tb+\inf_{x\in\D}(c+G^T\lambda+A^T\nu)^Tx\\
                          &=
              \begin{cases}
                -\lambda^Th-\nu^Tb, &c+G^T\lambda+A^T\nu=0\\
                -\infty, &\text{otherwise}
              \end{cases}
          \end{align*}
    \item Lagrange dual problem
          \begin{align*}
            \text{maximize\;}\quad& g(\lambda,\nu)=
              \begin{cases}
                -\lambda^Th-\nu^Tb, &c+G^T\lambda+A^T\nu=0\\
                -\infty, &\text{otherwise}
              \end{cases}\\
            \text{subject to}\quad& \lambda\succeq 0
          \end{align*}
    \item Make dual constraints explicit
    \begin{align*}
      \text{maximize\;}\quad& -\lambda^Th-\nu^Tb\\
      \text{subject to}\quad& c+G^T\lambda+A^T\nu=0\\
                            & \lambda\succeq 0
    \end{align*}
  \end{enumerate}
\end{proof}

% Exercise 4
\begin{exercise}
  Derive a dual problem for
  \begin{align*}
    \text{minimize\;\;}\quad& -\sum_{i=1}^m\log(b_i-a_i^Tx)
  \end{align*}
  with domain $\{x\mid a_i^Tx<b_i,\;i=1,\dots,m\}$.
  First introduce new variables $y_i$ and equality constraints $y_i=b_i-a_i^Tx$. (20\%)
\end{exercise}
\begin{proof}[Solution]
  \let\qed\relax
  The original problem is equivalent to
  \begin{align*}
    \text{minimize\;\;}\quad& -\sum_{i=1}^m\log y_i\\
    \text{subject to}  \quad& Ax+y-b=0
  \end{align*}
  where $x$ and $y$ are variables, and $A=[a_1\;a_2\;\cdots\;a_m]^T\in\R^{m\times n}$\\
  The Lagrange is
  \begin{align*}
    L(x,y,\nu)&=f_0(x)+\sum_{i=1}^p\nu_ih_i(x)\\
              &=-\sum_{i=1}^m\log y_i+\nu^T(Ax+y-b)
  \end{align*}
  The Lagrange dual function is
  \begin{align*}
    g(\nu)&=\inf_{x,y\in\D}L(x,y,\nu)\\
          &=\inf_{x,y\in\D}\left(-\sum_{i=1}^m\log y_i+\nu^T(Ax+y-b)\right)\\
          &=
      \begin{cases}
        \sum_{i=1}^m\log \nu_i+m-b^T\nu,&A^T\nu=0,\;\nu\succ 0\\
        -\infty,&\text{otherwise}
      \end{cases}
  \end{align*}
  and we use these properties to compute the result
  \begin{enumerate}[label=(\roman*)]
    \item $\nu^TAx$ is unbounded below, or is zero when $A^T\nu=0$
    \item since $y\succ 0$ is the domain of $y$, $\nu^Ty$ is unbounded below if $\nu\nsucc 0$
    \item by analysis the derivative of terms in $y$, we know that it achieves the minimum at $y_i=1/\nu_i$, \ie $-\sum_{i=1}^m\log y_i+\nu^Ty=\sum_{i=1}^m\log \nu_i+m$
  \end{enumerate}
  The Lagrange dual problem is
  \begin{align*}
    \text{maximize\;}\quad& \sum_{i=1}^m\log \nu_i+m-b^T\nu\\
    \text{subject to}\quad& A^T\nu=0\\
                          & \nu\succ 0
  \end{align*}
\end{proof}

\end{document}
