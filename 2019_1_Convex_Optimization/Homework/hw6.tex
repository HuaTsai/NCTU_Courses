%!TEX program = pdflatex
\documentclass[12pt]{extarticle}
\usepackage[english]{babel}
\usepackage{graphicx}
\usepackage{framed}
\usepackage[normalem]{ulem}
\usepackage{amsmath}
\usepackage{amsthm}
\usepackage{amssymb}
\usepackage{amsfonts}
\usepackage{enumerate}
\usepackage{enumitem}
\usepackage[utf8]{inputenc}
\usepackage[top=1in,bottom=1in,left=1in,right=1in]{geometry}
\usepackage{mdframed}
\usepackage{xcolor}
\usepackage{listings}
\usepackage[numbered,framed]{matlab-prettifier}
\usepackage{hyperref}

\hypersetup{
  colorlinks=true,
  linkcolor=blue
}
\theoremstyle{definition}
\newtheorem{exercise}{Exercise}
\everymath{\displaystyle}

\newcommand{\ones}{\mathbf 1}
\newcommand{\E}{\mathop{\bf E{}}}          % not used now
\newcommand{\R}{\mathbb{R}}
\newcommand{\Z}{\mathbb{Z}}
\newcommand{\D}{\mathcal{D}}
\newcommand{\symm}{\mathbb{S}}
\newcommand{\<}{\langle}
\renewcommand{\>}{\rangle}

\newcommand{\nullspace}{{\mathcal N}}      % not used now
\newcommand{\range}{{\mathcal R}}          % not used now
\newcommand{\Rank}{\mathop{\bf Rank}}      % not used now
\newcommand{\tr}{\mathop{\bf tr}}
\newcommand{\diag}{\mathop{\bf diag}}
\newcommand{\card}{\mathop{\bf card}}      % not used now
\newcommand{\rank}{\mathop{\bf rank}}      % not used now
\newcommand{\aff}{\mathop{\bf aff}}
\newcommand{\conv}{\mathop{\bf conv}}
\newcommand{\prox}{\mathbf{prox}}
\newcommand{\prob}{\mathop{\bf prob}}      % not used now
\newcommand{\dist}{\mathop{\bf dist{}}}
\newcommand{\argmin}{\mathop{\rm argmin}}  % not used now
\newcommand{\argmax}{\mathop{\rm argmax}}  % not used now
\newcommand{\epi}{\mathop{\bf epi}}
\newcommand{\Vol}{\mathop{\bf vol}}        % not used now
\newcommand{\dom}{\mathop{\bf dom}}
\newcommand{\intr}{\mathop{\bf int}}
\newcommand{\bd}{\mathop{\bf bd}}
\newcommand{\relintr}{\mathop{\bf relint}}
\newcommand{\cl}{\mathop{\bf cl}}
\newcommand{\sign}{\mathop{\bf sign}}

\newcommand{\cf}{\textit{cf.}}
\newcommand{\eg}{\textit{e.g.}, }
\newcommand{\ie}{\textit{i.e.}, }
\newcommand{\etal}{\textit{et al}. }
\newcommand{\etc}{\textit{etc.}}


\title{Assignment \#6}
\author{Shao Hua, Huang 0750727\\
ECM5901 - Optimization Theory and Application}
\begin{document}
\maketitle

% Exercise 1
\begin{exercise}
  Derive a dual problem for
  \begin{align*}
    \text{minimize}\quad \sum_{i=1}^N\|A_ix-b_i\|_2+\frac{1}{2}\|x-x_0\|_2^2
  \end{align*}
  The problem data are $A_i\in \R^{m_i\times n},\;b_i\in\R^{m_i}$, and $x_0\in\R^n$.
  First introduce new variables $y_i\in\R^{m_i}$ and equality constraints $y_i=A_ix-b_i$. (20\%)
\end{exercise}
\begin{proof}[Solution]
  \let\qed\relax
  The original problem is equivalent to
  \begin{align*}
    \text{minimize\;\;}\quad& \sum_{i=1}^N\|y_i\|_2+\frac{1}{2}\|x-x_0\|_2^2\\
    \text{subject to}  \quad& y_i=A_ix-b_i,\;i=1,\dots,N
  \end{align*}
  The Lagrange is
  \begin{align*}
    L(x,z_1,\dots,z_N)=\sum_{i=1}^N\|y_i\|_2+\frac{1}{2}\|x-x_0\|_2^2+\sum_{i=1}^Nz_i^T(y_i-A_ix+b_i)
  \end{align*}
\end{proof}

% Exercise 2
\begin{exercise}
  (A convex problem in which strong duality fails.)
  Consider the optimization problem
  \begin{align*}
    \text{minimize\;\;}\quad&e^{-x}\\
    \text{subject to}  \quad&x^2/y\le 0
  \end{align*}
  with variables $x$ and $y$, and domain $\D=\{(x,y)\mid y>0\}$.
  \begin{enumerate}[label=(\alph*)]
    \item Verify that this is a convex optimization problem. Find the optimal value. (5\%)
    \item Give the Lagrange dual problem, and find the optimal solution $\lambda^\ast$ and optimal value $d^\ast$ of the dual problem. (10\%)
    \item What is the optimal duality gap? Does Slater's condition hold for this problem? (5\%)
  \end{enumerate}
\end{exercise}
\begin{proof}[Solution]
  \let\qed\relax
  $ $
  \begin{enumerate}[label=(\alph*)]
    \item a
    \item b
    \item c
  \end{enumerate}
\end{proof}

% Exercise 3
\begin{exercise}
  Prove (without using any linear programming code) that the optimal solution of the LP
  \begin{align*}
    \text{minimize\;\;}\quad&47x_1+93x_2+17x_3-93x_4\\
    \text{subject to}  \quad&
      \begin{bmatrix}
        -1 & -6 & 1 & 3\\
        -1 & -2 & 7 & 1\\
        0 & 3 & -10 & -1\\
        -6 & -11 & -2 & 12\\
        1 & 6 & -1 & -3
      \end{bmatrix}
      \begin{bmatrix}
        x_1\\x_2\\x_3\\x_4
      \end{bmatrix}\preceq
      \begin{bmatrix}
        -3\\5\\-8\\-7\\4
      \end{bmatrix}
  \end{align*}
  is unique, and given by $x^\ast=(1,1,1,1)$. (20\%)
\end{exercise}
\begin{proof}
\end{proof}

% Exercise 4
\begin{exercise}
  (SDP relaxations of two-way partitioning problem).
  We consider the two-way partitioning problem (5.7), described on page 219,
  \begin{align}
    \text{minimize\;\;}\quad& x^TWx\nonumber\\
    \text{subject to}  \quad& x_i^2=1,\;i=1,\dots,n\label{eq:4.1}
  \end{align}
  with variable $x\in\R^n$.
  The Lagrange dual of this (nonconvex) problem is given by the SDP
  \begin{align}
    \text{maximize\;}\quad& -\ones^T\nu\nonumber\\
    \text{subject to}\quad& W+\diag(\nu)\succeq 0\label{eq:4.2}
  \end{align}
  with variable $\nu\in\R^n$.
  The optimal value of this SDP gives a lower bound on the optimal value of the partitioning problem \eqref{eq:4.1}.
  In this exercise we derive another SDP that gives a lower bound on the optimal value of the two-way partitioning problem, and explore the connection between the two SDPs.
  \begin{enumerate}[label=(\alph*)]
    \item Two-way partitioning problem in matrix form.
          Show that the two-way partitioning problem can be cast as
          \begin{align*}
            \text{maximize\;}\quad& \tr(WX)\\
            \text{subject to}\quad& X\succeq 0,\;\rank(X)=1\\
                                  & X_{ii}=1,\;i=1,\dots,n
          \end{align*}
          with variable $X\in\symm^n$.
          Hint.
          Show that if $X$ is feasible, then it has the form $X=xx^T$, where $x\in\R^n$ satisfies $x_i\in\{+1,-1\}$ (and vice versa). (5\%)
    \item (SDP relaxation of two-way partitioning problem.) 
          Using the formulation in part (a), we can form the relaxation
          \begin{align}
            \text{maximize\;}\quad& \tr(WX)\nonumber\\
            \text{subject to}\quad& X\succeq 0\label{eq:4.3}\\
                                  & X_{ii}=1,\;i=1,\dots,n\nonumber
          \end{align}
          with variable $X\in\symm^n$.
          This problem is an SDP, and therefore can be solved efficiently.
          Explain why its optimal value gives a lower bound on the optimal value of the two-way partitioning problem \eqref{eq:4.1}.
          What can you say if an optimal point $X^\ast$ for this SDP has rank one? (5\%)
    \item We now have two SDPs that give a lower bound on the optimal value of the two-way partitioning problem \eqref{eq:4.1}:
          the SDP relaxation \eqref{eq:4.3} found in part (b),
          and the Lagrange dual of the two-way partitioning problem, given in \eqref{eq:4.2}.
          What is the relation between the two SDPs?
          What can you say about the lower bounds found by them?
          Hint: Relate the two SDPs via duality. (10\%)
  \end{enumerate}
\end{exercise}
\begin{proof}
  $ $
  \begin{enumerate}[label=(\alph*)]
    \item a
    \item b
    \item c
  \end{enumerate}
\end{proof}

% Exercise 5
\begin{exercise}
  The pure Newton method.
  Newton's method with fixed step size $t=1$ can diverge if the initial point is not close to $x^\ast$.
  Consider
  \begin{align*}
    \text{minimize}\quad f(x)=\log(e^x+e^{-x})
  \end{align*}
  $f(x)$ has a unique minimizer $x^\ast=0$.
  Run Newton's method with fixed step size $t=1$, starting at $x^{(0)}=1$ and at $x^{(0)}=1.2$.
  Show the errors of the first four iterates.
  (You can do it by hand or using \textsc{Matlab}.)
\end{exercise}
\begin{proof}[Solution]
  \let\qed\relax
  $ $
  \begin{lstlisting}[style=Matlab-editor]
    for ok
    while ok
  \end{lstlisting}
\end{proof}


\end{document}
