\chapter{Convex functions}

% 3.1
\section{Basic properties and examples}

% 3.1.1
\subsection{Definition}
A fuction $f:\R^n\rightarrow\R$ is \textit{convex} if $\dom f$ is convex and
\begin{align}
  f(\theta x+(1-\theta)y)\le \theta f(x)+(1-\theta)f(y)\quad\forall x,y\in\dom f,\;0\le\theta\le 1\label{eq:3.1}
\end{align}
\begin{itemize}
  \item \textit{strictly convex}: strictly inequality holds in \eqref{eq:3.1} when $x\neq y$ and $0<\theta<1$
  \item \textit{concave}: $-f$ is convex
  \item \textit{strictly concave}: $-f$ is strictly convex
  \item affine function is both convex and concave
\end{itemize}
A function is convex if and only if it is convex when restricted to any line that intersects its domain. \ie
\begin{align*}
  f\text{ is convex}\Leftrightarrow\forall x\in\dom f,\;\forall v,\;g(t)=f(x+tv)\text{ is convex}
\end{align*}
on its domain $\{t\mid x+tv\in\dom f\}$.

% 3.1.2
\subsection{Extended-value extensions}
If $f$ is convex, define \textit{extended-value extension} $\tilde{f}:\R^n\rightarrow\R\cup\{\infty\}$ by
\begin{align*}
  \tilde{f}(x)=
    \begin{cases}
      f(x)\quad&x\in\dom f\\
      \infty\quad&x\notin\dom f
    \end{cases}
\end{align*}
\begin{itemize}
  \item recover original domain as $\dom f=\{x\mid \tilde{f}(x)<\infty\}$
  \item express \eqref{eq:3.1} with ``\textit{any} $x$ and $y$"
  \item for $f=f_1+f_2$, we need write $\dom f$ but need not write $\dom \tilde{f}$ explicitly
  \item the rest of this book implicitly use extended-value extensions
  \item for extended-value extension of a concave function, define $-\infty$
\end{itemize}
\begin{example}[\textit{Indicator function of a convex set}]
  Let $C\subseteq\R^n$ be a convex set and consider the convex function $I_C(x)=0\quad\forall x\in C$, we have
  \begin{align*}
    \tilde{I}_C(x)=
      \begin{cases}
        0\quad&x\in C\\
        \infty\quad&x\notin C
      \end{cases}
  \end{align*}
  The function $\tilde{I}_C$ is the \textit{indicator function} of the set $C$.\\
  Minimize $f$ ($\dom f=\R^n$) on $C$ is the same as minimize $f+\tilde{I}_C$ on $\R^n$
  \begin{itemize}
    \item indeed, $f+\tilde{I}_C$ is $f$ restricted to the set $C$
  \end{itemize}
\end{example}

% 3.1.3
\subsection{First-order conditions}
\label{subsec:3.1.3}
Suppose $f$ is differentiable (\ie $\nabla f$ exists on $\dom f$, $\dom f$ is open). Then $f$ is convex if and only if $\dom f$ is convex and
\begin{align}
  f(y)\ge f(x)+\nabla f(x)^T(y-x)\quad\forall x,y\in\dom f\label{eq:3.2}
\end{align}
\begin{itemize}
  \item convex function $\Leftrightarrow$ first-order Taylor approximation is a \textit{global underestimator}
  \item if $\nabla f(x)=0$, then $f(y)\ge f(x)\quad\forall y\in \dom f$, \ie $x$ is global minimizer of $f$
  \item strictly convex: $f(y)>f(x)+\nabla f(x)^T(y-x)\quad\forall x,y\in\dom f$
  \item for concave and strictly concave, change the inequality direction
\end{itemize}
\subsubsection{Proof of first-order convexity condition}
\begin{proof}
  \textbf{Add in future}
\end{proof}

% 3.1.4
\subsection{Second-order conditions}
\begin{example}[\textit{Quadratic functions}]
\end{example}
\begin{remark}
\end{remark}

% 3.1.5
\subsection{Examples}

% 3.1.6
\subsection{Sublevel sets}
\begin{example}
\end{example}

% 3.1.7
\subsection{Epigraph}
\begin{example}[\textit{Matrix fractional function}]
\end{example}

% 3.1.8
\subsection{Jensen's inequality and extensions}
\begin{remark}
\end{remark}

% 3.1.9
\subsection{Inequalities}

% 3.2
\section{Operations that preserve convexity}

% 3.2.1
\subsection{Nonnegative weighted sums}

% 3.2.2
\subsection{Composition with an affine mapping}

% 3.2.3
\subsection{Pointwise maximum and supremum}
\begin{example}[\textit{Piecewise-linear functions}]
\end{example}
\begin{example}[\textit{Sum of r largest components}]
\end{example}
\begin{example}[\textit{Support function of a set}]
\end{example}
\begin{example}[\textit{Distance to farthest point of a set}]
\end{example}
\begin{example}[\textit{Least-squares cost as a function of weights}]
\end{example}
\begin{example}[\textit{Maximum eigenvalue of a symmetric matrix}]
\end{example}
\begin{example}[\textit{Norm of a matrix}]
\end{example}

% 3.2.4
\subsection{Composition}
\begin{example}
\end{example}
\begin{example}[\textit{Simple composition results}]
\end{example}
\begin{remark}
\end{remark}
\begin{example}[\textit{Vector composition examples}]
\end{example}

% 3.2.5
\subsection{Minimization}
\begin{example}[\textit{Schur complement}]
\end{example}
\begin{example}[\textit{Distance to a set}]
\end{example}
\begin{example}
\end{example}

% 3.2.6
\subsection{Perspective of a function}
\begin{example}[\textit{Euclidean norm squared}]
\end{example}
\begin{example}[\textit{Negative logarithm}]
\end{example}
\begin{example}
\end{example}

% 3.3
\section{The conjugate function}

% 3.3.1
\subsection{Definition and examples}
Let $f:\R^n\rightarrow\R$, then the \textit{conjugate} of $f$, denoted by $f^\ast$ is defined as
\begin{align}
  f^\ast(y)=\sup_{x\in\dom f}\{y^Tx-f(x)\}\label{eq:3.18}
\end{align}
\begin{itemize}
  \item $\dom f^\ast$ consists of $y\in\R^n$ for which supremum is finite, \ie $y^Tx-f(x)$ is bounded above on $\dom f$
  \item $f^\ast$ is a convex function, since it is the pointwise supremum of affine functions of $y$
  \item when $f$ is convex, the subscript $x\in\dom f$ is not necessary since $y^Tx-f(x)=-\infty$ for $x\notin\dom f$
\end{itemize}
\begin{example}
  A few of conjugate examples.
  \begin{itemize}
    \item \textit{Affine function.}
          $f(x)=ax+b$.
          The function $xy-(ax+b)$ is bounded at $\{a\}$.
          Therefore, $f^\ast(a)=-b$ with $\dom f^\ast=\{a\}$.
    \item \textit{Negative logarithm.}
          $f(x)=-\log(x),\;\dom f=\R_{++}$.
          The function $xy+log(x)$ is unbounded above if $y\ge 0$, and maximum if $y<0,;x=-1/y$.
          Therefore, $f^\ast(y)=-\log(-y)-1$ with $\dom f^\ast=-\R_{++}$.
    \item \textit{Exponential.}
          $f(x)=e^x$.
          The function $xy-e^x$ is unbounded if $y\le 0$, maximum if $y>0,\;x=\log(y)$, and maximum $0$ if $y=0$.
          Therefore, $f^\ast(y)=y\log(y)-y$ with $\dom f^\ast=\R_{+}$.
    \item \textit{Negative entropy.}
          $f(x)=x\log(x),\;\dom f=\R_{+},\;f(0)=0$.
          The function $xy-x\log(x)$ is bounded above for all $y$, and attains maximum at $x=e^{y-1}$.
          Therefore, $f^\ast(y)=e^{y-1}$ with $\dom f^\ast=\R$.
    \item \textit{Inverse.}
          $f(x)=1/x,\;\dom f=\R_{++}$.
          The function $xy-1/x$ is unbounded above if $y>0$, maximum if $y<0,\;x=(-y)^{-1/2}$, and maximum $0$ if $y=0$.
          Therefore, $f^\ast(y)=-2(-y)^{1/2}$ with $\dom f^\ast=-R_{+}$.
  \end{itemize}
\end{example}
\begin{example}[\textit{Strictly convex quadratic function}]
\end{example}
\begin{example}[\textit{Log-determinant}]
\end{example}
\begin{example}[\textit{Indicator function}]
\end{example}
\begin{example}[\textit{Log-sum-exp function}]
\end{example}
\begin{example}[\textit{Norm}]
\end{example}
\begin{example}[\textit{Norm squared}]
\end{example}
\begin{example}[\textit{Revenue and profit functions}]
\end{example}

% 3.3.2
\subsection{Basic properties}

% 3.4
\section{Quasiconvex functions}

% 3.4.1
\subsection{Definition and examples}
\begin{example}
\end{example}
\begin{example}[\textit{Length of a vector}]
\end{example}
\begin{example}
\end{example}
\begin{example}[\textit{Linear-fractional function}]
\end{example}
\begin{example}[\textit{Distance ratio function}]
\end{example}
\begin{example}[\textit{Internal rate of return}]
\end{example}

% 3.4.2
\subsection{Basic properties}
\begin{example}[\textit{Cardinality of a nonegative vector}]
\end{example}
\begin{example}[\textit{Rank of positive semidefinite matrix}]
\end{example}

% 3.4.3
\subsection{Differentiable quasiconvex functions}

% 3.4.4
\subsection{Operations that preserve quasiconvexity}
\begin{example}[\textit{Generalized eigenvalue}]
\end{example}

% 3.4.5
\subsection{Representation via family of convex functions}
\begin{example}[\textit{Convex over concave function}]
\end{example}

% 3.5
\section{Log-concave and log-convex functions}

% 3.5.1
\subsection{Definition}
A function $f:\R^n\rightarrow \R$ is \textit{logarithmically concave} or \textit{log-concave} if $f(x)>0$ for all $x\in\dom f$ and $\log f$ is concave.
Similarly, $f$ is \textit{logarithmically convex} or \textit{log-convex} if $\log f$ is convex.
\begin{itemize}
  \item $f$ is log-convex if and only if $1/f$ is log-concave
  \item it is convenient to allow $f$ to take on the value zero with $\log f(x)=-\infty$
\end{itemize}\par
A function $f:\R^n\rightarrow \R$ with convex domain and $f(x)>0$ for all $x\in\dom f$, is log-concave if and only if
\begin{align*}
  f(\theta x+(1-\theta)y)\ge f(x)^\theta f(y)^{1-\theta}\quad\forall x,y\in\dom f
\end{align*}
where $0\le\theta\le 1$.
\begin{example}[\textit{Some simple examples of log-concave and log-convex functions}]
\end{example}
\begin{example}[\textit{Log-concave density functions}]
\end{example}

% 3.5.2
\subsection{Properties}
\subsubsection{Twice differentiable log-convex/concave functions}
Suppose $f$ is twice differentiable, with $\dom f$ convex, so
\begin{align*}
  \nabla^2 \log f(x)=\frac{1}{f(x)}\nabla^2 f(x)-\frac{1}{f(x)^2}\nabla f(x)\nabla f(x)^T
\end{align*}
Hence $f$ is log-convex if and only if for all $x\in\dom f$
\begin{align*}
  f(x)\nabla^2 f(x)\succeq\nabla f(x)\nabla f(x)^T
\end{align*}
and log-concave if and only if for all $x\in\dom f$
\begin{align*}
  f(x)\nabla^2 f(x)\preceq\nabla f(x)\nabla f(x)^T
\end{align*}
\begin{example}[\textit{Laplace transform of a nonnegative function and the moment and cumulant generating functions}]
\end{example}
\begin{example}
\end{example}
\begin{example}[\textit{Yield function}]
\end{example}
\begin{example}[\textit{Volume of polyhedron}]
\end{example}

% 3.6
\section{Convexity with respect to generalized inequalities}

% 3.6.1
\subsection{Monotonicity with respect to a generalized inequality}
\begin{example}[\textit{Monotone vector functions}]
\end{example}
\begin{example}[\textit{Matrix monotone functions}]
\end{example}

% 3.6.2
\subsection{Convexity with respect to a generalized inequality}
\begin{example}[\textit{Convexity with respect to componentwise inequality}]
\end{example}
\begin{example}[\textit{Matrix convexity}]
\end{example}
\begin{example}
\end{example}
