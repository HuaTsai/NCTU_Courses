\chapter{Convex sets}

% 2.1
\section{Affine and convex sets}

% 2.1.1
\subsection{Lines and line segments}
A \textit{line} passing through $x_1$ and $x_2$ on $\R^n$ ($x_1\ne x_2$) has the form
\begin{align*}
  y &= \theta x_1+(1-\theta)x_2\\
    &= x_2+\theta(x_1-x_2)
\end{align*}
It becomes \textit{line segment} when $0\le\theta\le1$

% 2.1.2
\subsection{Affine sets}
Set $C\subseteq\R^n$ is \textit{affine} if the line through any two distinct points in $C$ lies in $C$, \ie\\
for any $x_1, x_2\in C$ and $\theta\in\R$, we have $\theta x_1+(1-\theta)x_2\in C$\\
The \textit{affine combination} of $x_1,\dots,x_k$ is
\begin{align*}
  \theta_1x_1+\theta_2x_2+\dots+\theta_kx_k\quad\text{where}\quad\theta_1+\theta_2+\dots+\theta_k=1
\end{align*}
If $C\subseteq \R^n$ is affine and $x_1,\dots,x_k\in C$, then the affine combination of $x_1,\dots,x_k$ is in $C$.
This property can be proved by induction.\par
If $C$ is affine and $x_0\in C$, then $V=C-x_0=\{x-x_0\mid x\in C\}$ is a subspace.
\begin{proof}
  let $v_1=x_1-x_0\in V$ and $v_2=x_2-x_0\in V$
  \begin{align*}
    \alpha v_1+\beta v_2
      &= \alpha (x_1-x_0)+\beta (x_2-x_0)\\
      &= \alpha x_1+\beta x_2+(1-\alpha-\beta)x_0-x_0\in V
  \end{align*}
  since $\alpha x_1+\beta x_2+(1-\alpha-\beta)x_0\in C$.
  Therefore, $V$ is a subspace.
\end{proof}
\begin{itemize}
  \item subspace $V$ associated with an affine set $C$ does not depend on choice of $x_0$
  \item definition of the \textit{dimension} of an affine set $C$ is the dimension of subspace $V=C-x_0$
\end{itemize}
\begin{example}[\textit{Solution set of linear equations}]
  The solution set of $C=\{x\mid Ax=b\}$ is an affine set. Let $x_1,x_2\in C$, then $Ax_1=A_2=b$
  \begin{align*}
    A(\theta x_1+(1-\theta)x_2)=\theta Ax_1+(1-\theta)Ax_2=b
  \end{align*}
  \begin{itemize}
    \item the subspace associated with the affine set $C$ is the nullspace of $A$
    \item converse: every affine set can be expressed as the solution set of a system of linear equations
  \end{itemize}
\end{example}
The \textit{affine hull} of $C$ is the set of all affine combinations
\begin{align*}
  \aff C=\{\theta_1x_1+\cdots+\theta_kx_k\mid x_1,\dots,x_k\in C,\;\sum_{i=1}^k\theta_i=1\}
\end{align*}
\begin{itemize}
  \item affine hull is the smallest affine set that contains $C$
  \item if $S$ is affine set with $C\subseteq S$, then $\aff C\subseteq S$
\end{itemize}

% 2.1.3
\subsection{Affine dimension and relative interior}
\begin{example}
\end{example}

% 2.1.4
\subsection{Convex sets}

% 2.1.5
\subsection{Cones}

% 2.2
\section{Some important examples}

% 2.2.1
\subsection{Hyperplanes and halfspaces}

% 2.2.2
\subsection{Euclidean balls and ellipsoids}

% 2.2.3
\subsection{Norm balls and norm cones}
\begin{example}[\textit{Second-order cone}]
\end{example}

% 2.2.4
\subsection{Polyhedra}
\begin{example}[\textit{Nonnegative orthant}]
\end{example}
\begin{example}[\textit{Some common simplexes}]
\end{example}

% 2.2.5
\subsection{The positive semidefinite cone}
\begin{example}[\textit{Positive semidefinite cone in $\symm^2$}]
\end{example}

% 2.3
\section{Operations that preserve convexity}

% 2.3.1
\subsection{Intersection}
\begin{example}
\end{example}
\begin{example}
\end{example}

% 2.3.2
\subsection{Affine functions}
\begin{example}[\textit{Polyhedron}]
\end{example}
\begin{example}[\textit{Solution set of linear matrix inequality}]
\end{example}
\begin{example}[\textit{Hyperbolic cone}]
\end{example}
\begin{example}[\textit{Ellipsoid}]
\end{example}

% 2.3.3
\subsection{Linear-fractional and perspective functions}
\begin{remark}
\end{remark}
\begin{remark}
\end{remark}
\begin{example}[\textit{Conditional probabilities}]
\end{example}

% 2.4
\section{Generalized inequalities}

% 2.4.1
\subsection{Proper cones and generalized inequalities}
\begin{example}[\textit{Nonnegative orthant and componentwise inequality}]
\end{example}
\begin{example}[\textit{Positive semidefinite cone and matrix inequality}]
\end{example}
\begin{example}[\textit{Cone of polynomials nonegative on $[0,1]$}]
\end{example}

% 2.4.2
\subsection{Minimum and minimal elements}
\begin{example}
\end{example}
\begin{example}[\textit{Minimum and minimal elements of a set of symmetric matrices}]
\end{example}

% 2.5
\section{Separating and supporting hyperplanes}

% 2.5.1
\subsection{Separating hyperplane theorem}
\begin{example}[\textit{Separation of an affine and a convex set}]
\end{example}
\begin{example}[\textit{Strict separation of a point and a closed convex set}]
\end{example}
\begin{example}[\textit{Theorem of alternatives for strict linear inequalities}]
\end{example}

% 2.5.2
\subsection{Supporting hyperplanes}

% 2.6
\section{Dual cones and generalized inequalities}

% 2.6.1
\subsection{Dual cones}
The \textit{dual cone} of a cone $K$ is defined by
\begin{align}
  K^\ast=\{y\mid x^Ty\ge 0,\;\forall x\in K\}
\end{align}
\begin{itemize}
  \item dual cone is a cone
  \item dual cone is always convex (even if origin cone is not)
  \item $y\in K^\ast$ if and only if $-y$ is the normal of a hyperplane supports $K$ at the origin
\end{itemize}
\begin{example}[\textit{Subspace}]
  The dual cone of a subspace $V\subseteq \R^n$ is its orthogonal complement $V^\perp=\{y\mid y^Tv=0,\;\forall v\in V\}.$
  \begin{proof}
    $ $
    \begin{enumerate}[label=(\roman*)]
      \item trivially, $V^\perp\subseteq V^\ast$
      \item start here
    \end{enumerate}
  \end{proof}
\end{example}
\begin{example}[\textit{Nonnegative orthant}]

\end{example}
\begin{example}[\textit{Positive semidefinite cone}]
\end{example}
\begin{example}[\textit{Dual of a norm cone}]
\end{example}

% 2.6.2
\subsection{Dual generalized inequalities}
\begin{example}[\textit{Theorem of alternatives for linear strict generalized inequalities}]
\end{example}

% 2.6.3
\subsection{Minimum and minimal elements via dual inequalities}
\begin{example}[\textit{Pareto optimal production frontier}]
\end{example}
