\chapter{Mathematical background}

\section{Norms}

\subsection{Inner product, Euclidean norm, and angle}
Define \textit{standard inner product} on $\R^n$ by
\begin{align*}
  \<x,y\>=x^Ty=\sum_{i=1}^nx_iy_i
\end{align*}
Define \textit{Euclidean norm}, $\ell_2$-norm on $x\in\R^n$ by
\begin{align}
  \|x\|_2=(x^Tx)^{1/2}=(x_1^2+\dots+x_n^2)^{1/2}
\end{align}
The \textit{Cauchy-Schwartz inequality}
\begin{align*}
  |x^Ty|\le \|x\|_2\|y\|_2
\end{align*}
Define \textit{standard inner product} on $\R^{m\times n}$
\begin{align*}
  \<X,Y\>=\tr(X^TY)=\sum_{i=1}^m\sum_{j=1}^nX_{ij}Y_{ij}
\end{align*}

\subsection{Norms, distance, and unit ball}
A function $f:\R^n\rightarrow\R$ with $\dom f=\R^n$ is a norm if
\begin{itemize}
  \item $f$ is nonnegative: $f(x)\ge 0\quad\forall x\in R^n$
  \item $f$ is definite: $x=0 \Rightarrow f(x)=0$
  \item $f$ is homogeneous: $f(tx)=|t|f(x),\quad\forall x\in\R^n$ and $t\in\R$
  \item $f$ satisfies the triangle inequality: $f(x+y)\le f(x)+f(y), \quad\forall x, y \in \R^n$
\end{itemize}

\subsection{Examples}

Define \textit{Chebyshev norm}, $\ell_\infty$-norm by
\begin{align*}
  \|x\|_\infty=max\{|x_1|,\dots.|x_n|\}
\end{align*}

\subsection{Equivalence of norms}

\subsection{Operator norms}
Suppose $\|\cdot\|_a$ and $\|\cdot\|_b$ are norms on $\R^m$ and $\R^n$.
Define \textit{operator norm} of $X\in\R^{m\times n}$, induced by $\|\cdot\|_a$ and $\|\cdot\|_b$, as
\begin{align*}
  \|X\|_{a,b}=\sup\{\|Xu\|_a\mid\|u\|_b\le 1\}.
\end{align*}
The \textit{spectral norm}/$\ell_2$\textit{-norm} of $X$ is its \textit{maximum singular value}
\begin{align*}
  \|X\|_2=\sigma_\text{max}(X)=(\lambda_\text{max}(X^TX))^{1/2}.
\end{align*}
The $\ell_\infty$\textit{-norm} of $X$ is the \textit{max-row-sum norm}
\begin{align*}
  \|X\|_\infty=\sup\{\|Xu\|_\infty\mid\|u\|_\infty\le 1\}=\max_{i=1,\dots,m}\sum_{j=1}^n|X_{ij}|.
\end{align*}
The $\ell_1$\textit{-norm} of $X$ is the \textit{max-column-sum norm}
\begin{align*}
  \|X\|_1=\max_{j=1,\dots,n}\sum_{i=1}^m|X_{ij}|.
\end{align*}

\subsection{Dual norm}
Let $\|\cdot\|$ be a norm on $\R^n$, then the associated \textit{dual norm} is
\begin{align*}
  \|z\|_\ast=\sup\{z^Tx\mid\|x\|\le 1\}
\end{align*}

\section{Analysis}

\subsection{Open and closed sets}

\subsection{Supremum and infimum}

\section{Functions}

\subsection{Function notation}


\subsection{Continuity}

\subsection{Closed functions}


\section{Derivatives}


\subsection{Derivative and gradient}

\subsection{Chain rule}

\subsection{Second derivative}



\subsection{Chain rule for second derivative}

\section{Linear algebra}

\subsection{Range and nullspace}

\subsection{Symmetric eigenvalue decomposition}


\subsection{Generalized eigenvalue decomposition}

\subsection{Singular value decomposition}

\subsection{Schur complement}
